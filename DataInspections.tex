
	\documentclass[a4paper,12pt]{article}
%%%%%%%%%%%%%%%%%%%%%%%%%%%%%%%%%%%%%%%%%%%%%%%%%%%%%%%%%%%%%%%%%%%%%%%%%%%%%%%%%%%%%%%%%%%%%%%%%%%%%%%%%%%%%%%%%%%%%%%%%%%%%%%%%%%%%%%%%%%%%%%%%%%%%%%%%%%%%%%%%%%%%%%%%%%%%%%%%%%%%%%%%%%%%%%%%%%%%%%%%%%%%%%%%%%%%%%%%%%%%%%%%%%%%%%%%%%%%%%%%%%%%%%%%%%%
\usepackage{eurosym}
\usepackage{vmargin}
\usepackage{amsmath}
\usepackage{graphics}
\usepackage{epsfig}
\usepackage{framed}
\usepackage{subfigure}
\usepackage{enumerate}
\usepackage{fancyhdr}

\setcounter{MaxMatrixCols}{10}
%TCIDATA{OutputFilter=LATEX.DLL}
%TCIDATA{Version=5.00.0.2570}
%TCIDATA{<META NAME="SaveForMode"CONTENT="1">}
%TCIDATA{LastRevised=Wednesday, February 23, 201113:24:34}
%TCIDATA{<META NAME="GraphicsSave" CONTENT="32">}
%TCIDATA{Language=American English}

\pagestyle{fancy}
\setmarginsrb{20mm}{0mm}{20mm}{25mm}{12mm}{11mm}{0mm}{11mm}
\lhead{Apache Spark} \rhead{Kevin O'Brien} \chead{Decision tree} %\input{tcilatex}

\begin{document}
%=========================================================================================%

\section{Inspecting a Data Set}
\begin{itemize}
	\item \texttt{dim()}
	\item \texttt{nrow()} and \texttt{ncol()}
	\item \texttt{names()}
	\item summary()
	\item \texttt{tail()}
	\item \texttt{head()}
	\item \texttt{describe()} (from the psych package)
\end{itemize}
%=========================================================================================%
\subsection*{3.1 Dimensions of a data set}
We have remarked that some data sets are very large. This is perhaps a good place to consider
summary information about data objects. 

For a simple vector, a useful command to determine
the length (remark: sample size) is the function \texttt{length()}.
%================================%
\begin{framed}
\begin{verbatim}
> Y=4:18
> length(Y)
[1] 15
\end{verbatim}
\end{framed}
For more complex data sets ( such as data frames,  which we will see a lot of later) , we have several
tools for assessing the size of data.
%================================%
\begin{framed}
\begin{verbatim}
> dim(iris) # dimensions of data set
[1] 150 5
> nrow(iris) # number of rows
[1] 150
> ncol(iris) # number of columns
[1] 5
11

\end{verbatim}
\end{framed}

We can also determine the row names and column names using the functions \texttt{row\texttt{names()}}
and \texttt{colnames()}. If there are no specific row or column names, the command will just return
the indices.
\begin{framed}
\begin{verbatim}
> colnames(iris)
[1] "Sepal.Length" "Sepal.Width" "Petal.Length" "Petal.Width" "Species"

\end{verbatim}
\end{framed}
%=========================================================================================%
\subsection*{3.2 The summary() command}
The command \texttt{summary()} is one of the most useful commands in R. It is a generic function used
to produce result summaries of the results of various functions. The function invokes particular
methods which depend on the class of the first argument. In other words, R picks out the most
suitable type of summary for that data.
\begin{framed}
\begin{verbatim}
> summary(iris)
Sepal.Length Sepal.Width Petal.Length Petal.Width Species
Min. :4.300 Min. :2.000 Min. :1.000 Min. :0.100 setosa :50
1st Qu.:5.100 1st Qu.:2.800 1st Qu.:1.600 1st Qu.:0.300 versicolor:50
Median :5.800 Median :3.000 Median :4.350 Median :1.300 virginica :50
Mean :5.843 Mean :3.057 Mean :3.758 Mean :1.199
3rd Qu.:6.400 3rd Qu.:3.300 3rd Qu.:5.100 3rd Qu.:1.800
Max. :7.900 Max. :4.400 Max. :6.900 Max. :2.500
>

\end{verbatim}
\end{framed}
\texttt{summary()} is particularly useful for manipulating data from more complex data objects.

%=========================================================================================%
\subsection{3.3 Structure of a Data Object}
The structure, class and storage mode of an object can be determined using the following
commands. Try out a few.
\begin{itemize}
	\item str()
	\item class()
	\item mode()
\end{itemize}
%----------------------------------------------- %
\begin{framed}
	\begin{verbatim}
	> class(Nile)
	[1] "ts"
	> mode(Nile)
	[1] "numeric"
	>
	> a
	[1] 6
	>
	> mode(a)
	[1] "numeric"
	> class(a)
	[1] "numeric"
	> str(a)
	num 6
	>
	> class(iris)
	[1] "data.frame"
	> mode(iris)
	[1] "list"
	\end{verbatim}
\end{framed}
%==================================================================================================%
\end{document}
