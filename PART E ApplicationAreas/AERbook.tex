%- http://www.springer.com/economics/econometrics/book/978-0-387-77316-2

This is the first book on applied econometrics using the R system for statistical computing and graphics. It presents hands-on examples for a wide range of econometric models, from classical linear regression models for cross-section, time series or panel data and the common non-linear models of microeconometrics such as logit, probit and tobit models, to recent semiparametric extensions. In addition, it provides a chapter on programming, including simulations, optimization, and an introduction to R tools enabling reproducible econometric research.
 
An R package accompanying this book, AER, is available from the Comprehensive R Archive Network (CRAN) at http://CRAN.R-project.org/package=AER.
 
It contains some 100 data sets taken from a wide variety of sources, the full source code for all examples used in the text plus further worked examples, e.g., from popular textbooks. The data sets are suitable for illustrating, among other things, the fitting of wage equations, growth regressions, hedonic regressions, dynamic regressions and time series models as well as models of labor force participation or the demand for health care.
 
The goal of this book is to provide a guide to R for users with a background in economics or the social sciences. Readers are assumed to have a background in basic statistics and econometrics at the undergraduate level. A large number of examples should make the book of interest to graduate students, researchers and practitioners alike.
 
Christian Kleiber is Professor of Econometrics and Statistics at Universität Basel, Switzerland. Achim Zeileis is Assistant Professor in the Dept. of Statistics and Mathematics at Wirtschaftsuniversität Wien, Austria. R users since version 0.64.0, they have been collaborating on econometric methodology in R, including several R packages, for the past eight years.
