
\subsection*{Packages}
%===========================================================================================%
\begin{frame}

Packages
"R"contains one or more libraries of packages. Packages contain
various functions and data sets for numerous purposes, e.g.
survival package, genetics package, fda package, etc.
Some packages are part of the basic installation. Others can be
downloaded from CRAN.
To access all of the functions and data sets in a particular package,
it must be loaded into the workspace. For example, to load the
fda package:
> library(fda)

One important thing to note is that if you terminate your session
and start a new session with the saved workspace, you must load
the packages again.

%===========================================================================================%
\begin{frame}
Packages
To check what packages are currently loaded into the workspace
> search()
[1] ".GlobalEnv" "package:MASS" "package:stats"
[4] "package:graphics" "package:grDevices" "package:utils"
[7] "package:datasets" "package:methods" "Autoloads"
[10] "package:base"

Can remove a package you have loaded use:
> detach("package:fda")

%===========================================================================================%
\begin{frame}

For the purposes of this session, a data set already stored in R will
be used. To access this data, must rst load the package
containing the data. (R has many packages containing various
functions that can be used to analyse data, e.g. if you want to
analyse your data using splines, need to load the splines
package). In this example, the data is stored in the MASS package.
This is loaded with the command
> library(MASS)
Now have access to all functions and data sets stored in this
package.


%===========================================================================================%
\begin{frame}
An interactive session
We will work with the data set titled \whiteside". To display the
data:
> whiteside
Insul Temp Gas
1 Before -0.8 7.2
2 Before -0.7 6.9
3 Before 0.4 6.4
4 Before 2.5 6.0
5 Before 2.9 5.8
6 Before 3.2 5.8
7 Before 3.6 5.6
8 Before 3.9 4.7
9 Before 4.2 5.8
10 Before 4.3 5.2
This is a particular type of object called a data frame.
A full description of these data is found using
> ? whiteside

%===========================================================================================%
\begin{frame}
An interactive session
To remind ourselves of the names of the columns:
> names(whiteside)
[1] "Insul" "Temp" "Gas"
Summary statistics for each column are determined using
> summary(whiteside)
Insul Temp Gas
Before:26 Min. :-0.800 Min. :1.300
After :30 1st Qu.: 3.050 1st Qu.:3.500
Median : 4.900 Median :3.950
Mean : 4.875 Mean :4.071
3rd Qu.: 7.125 3rd Qu.:4.625
Max. :10.200 Max. :7.200
Access the data in a particular column
> whiteside$Temp

%===========================================================================================%
\begin{frame}
An interactive session
A plot of gas consumption versus temperature is now created.
> plot(Gas ~ Temp, data=whiteside, pch=16)


\begin{frame}
Exercises I
Vectors and Factors
1. Create a vector x with the following entries
3 4 1 1 2 1 4 2 1 1 5 3 1 1 1 2 4 5 5 3
Check which elements of x are equal to 1 (Hint use ==
operator). Modify x so that all of the 1's are changed to 0's.
2. Create a vector y containing the elements of x that are
greater than 1.
3. Create a sequence of numbers from 1 to 20 in steps of 0.2
and store.
4. Concatenate x and y into a vector called newVec.
5. Display all objects in the workspace and then remove newVec
(see Lecture 1).

Exercises II
Vectors and Factors
6. Six patients were asked to rate their pain from 0 to 3, with 0
representing `no pain', 1 representing `mild' pain, 2
representing `medium' pain and 3 representing `severe'
pain. The following results were obtained:
Patient 1 2 3 4 5 6
Pain level 0 3 1 2 1 2
Create a factor fpain to represent the above data.

\end{frame}
%=========================================================%
\begin{frame}[fragile]
\Large
\frametitle{Exercise}

1. Create a data frame called \texttt{club.points} with the following
data.
Firstname Lastname Age Gender Points
Alice Ryan 37 F 278
Paul Collins 34 M 242
Jerry Burke 26 M 312
Thomas Dolan 72 M 740
Marguerite Black 18 F 177
Linda McGrath 24 F 195

\end{frame}
%=========================================================%
\begin{frame}[fragile]
\Large
\frametitle{Exercise}

\begin{verbatim}
Alice
Paul
Gerry
Thomas
Marguerite
\end{verbatim}

\end{frame}
%=========================================================%
\begin{frame}
\Large
\frametitle{Exercise}

2. Store the points for every person into a vector called pts,
then calculate the average number of points received. (Hint
use mean() function).

\end{frame}
%=========================================================%
\begin{frame}
\Large
\frametitle{Exercise}

3. Store the data for the females only into a data frame called
fpoints.
Exercises II
Data Frames
4. The age for Jerry Burke was entered incorrectly. Change his
age to 28.

\end{frame}
%=========================================================%
\begin{frame}
\Large
\frametitle{Exercise}

5. Determine the maximum age of the males.
6. Extract the data for people with more than 100 points and are
over the age of 30.
