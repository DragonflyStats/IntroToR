1. The R Environment

a) demo
b) script editor
c) working directory
getwd
 		setwd

2. Data Entry
a) scan
b) data entry spreadsheet style data input interface

3. Complex Numbers 
a) Im(), Re(), Conj()

4. Text parsing
a) Letters

5. Data manipulation
a) unique
if A is a vector or data frame, this command returns a similar object but with duplicate objects surpressed.


6. Set theory and sampling
a) sample sampling with or without replacement

7) Mathematics


Letters(4)
letters(5)

#############################################################
# The DEMO function 

demo() #  list of Demonstrations

demo(graphics) # demonstration of graphics functions


#############################################################
# The APROPOS function 

# Useful function for determining revelant commands or functions

# Suppose we want to find what inference tests are currently feasible

# Pick a short "string" that would match what you are searching for

apropos("test")

# Try this: Find a function relevant to using a) letters b) correlation


#############################################################

System Time and System Data


Time1 = Sys.time()
Time2 = Sys.time()

Time1-Time2
Time2-Time1




#############################################################

# The jitter functions - add some "noise" to a numeric value. Sometimes useful in graphics.

x = 2

jitter(x)

#############################################################

# Complex Number

X1 = 0 + 5i

X1	#print the complex number to screen

class(X1); str(X1)

Im(X1)		# The imaginary component

Re(X1)		# The Real component	

Conj(X1)	# The complex conjugate 

# Lets Mutliply two complex numbers

############################################################

# Using R for Categorical Data

# Two variables; nationality (NATS) and gender (GEND)

NATS = c("Irish","German","French","Irish","Spanish","Danish", "Swiss", "Greek", "Irish", "Irish","French")
GEND = c("Male","Female","Male","Female","Male","Female","Male","Male","Male","Female","Female")


#Check the number of items in each set

length(GEND)
length(NATS) 

# Produce a pair of Frequency Tables

table(GEND)
table(NATS)

# Produce a 2X2 Contingency Table

table(GEND,NATS)

############################################################

# The SCAN function

Vec = scan()

# Enter some numbers, hitting return each time
# Hit Return twice to complete input

# Print output to Screen

Vec 

# We will use this vector again later

############################################################

# RANK and ORDER functions

# using the vector from the SCAN() exercise

order(Vec)

	#Discuss the output for this function

rank(Vec)

	# Notice how it handles "ties"	


#closer look

rbind(Vec,rank(Vec))

#####################

#The WHICH function

which(Vec>4)
	
	#Discuss the output for this function
	#Contrast this with the ORDER function output


###########################################################

# Tukey's five number summary
# Returns the minimum, Q1, Median, Q3 and maximum of dataset
# Compare output to that of SUMMARY function. You may get different estimates for Quartiles


fivenum(Vec)
summary(Vec)

###########################################################

# Set Theory and Sampling

# Construct two vectors K and J

K = 1:6
J = 3:10

union (K,J)		#Union of J and K
intersect(K,J)		#Intersection of J and K	
			
setdiff(K,J)		#Elements of K that are not in J
setdiff(J,K)		#Elements of J that are not in K

# Are the two sets equal ( sometimes useful in programming)
# Output is TRUE or FALSE

setequal(K,J)

##########

# Sampling with replacement and without replacement

sample(K,4)

sample(K,7,replace=TRUE)



# A potential pitfall associated with the SAMPLE function

M = 9			#Vector that only one element

sample(M,1)

############################################################



############################################################



############################################################

sink data output function 
which returns a vector of which elements fulfil a certain condition

table calculates a frequency table for a vector


Set theory : union interesction set difference

cholesky decomposition
 chol ()
jacobi nethod 
decompose a matrix into D and R
fivenum tukeys five number summary

Exercise
slope intercept estimates through matrices

rbind
rev reverse the order of a vector
reverse
jitter

order values order() rank

apropos function

 useful from prograaming long simulations
sys.time()w
sys.date()

