
%=====================================================================================================================%
\begin{frame}

\frametitle{Introduction to R}

R is the product of an active movement among statisticians for a powerful, programmable, portable, and open computing environment, applicable to the most complex and sophsticated problems, as well as“routine”analysis.
 
There are no restrictions on access or use.


Statisticians have implemented hundreds of specialised statistical procedures for a wide variety of applications as contributed packages, which are also freely-available and which integrate directly into R.

 
R has data handling and storage facilities, a suite of operators for calculations on arrays in particular matrices. A large coherent integrated collection of intermediate tools for data analysis
A large selection of demonstration datasets used in the illustration of many statistical methods.
Graphical facilities for data analysis and display either directly.at the computes or on hardcopy.
 
\end{frame}
%=====================================================================================================================%
\begin{frame}
R is a flexible language that is object oriented and thus allows the manipulation of a complex data structures in a condensed and efficient manner.


R's graphical abilities are also remarkable with possible interfacing with text processors such as Latex with the package sweave.

R offers the addtional advantage of being a free and opensource system under the GNU general public licence.

R is primarily a statistical language. R can be installed free of charge from www.r-project.org
\end{frame}
%=====================================================================================================================%
\begin{frame}
An online guide "An Introduction to R" can be access by typing help.start() at the command prompt to access this.

R is a statistical Environment for statistical computing and graphics, which is available for windows, Unix and Mac OS platforms.

R is maintained and distributed by an international team of statisticians and computers scientists.
R is one of the major tools used in statistical research and in applications of statistics research.

R is an open-source (GPL) statistical environment modeled after S and S-Plus. The S language was developed in the late 1980s at AT&T labs. The R project was started by Robert Gentleman and Ross Ihaka of the Statistics Department of the University of Auckland in 1995. It has quickly gained a widespread audience. 
\end{frame}
%=====================================================================================================================%
\begin{frame}
R is currently maintained by the R core-development team, a hard-working, international team of volunteer developers. 

The R project web page ( http://www.r-project.org ) is the main site for information on R. At this site are directions for obtaining the software, accompanying packages and other sources of documentation.

\end{frame}

%=====================================================================================================================%
\begin{frame}

For example, to load the fda package:

> library(fda)

One important thing to note is that if you terminate your session and start a new session with the saved workspace, you must load  the packages again



install.packages("evir")
 
To get out of R, just type: q(). 

\end{frame}

\begin{frame}

Section 10: Simulation
Simulation Study : Random Walks
Simulation Study: Distribution of pairwise maxima and minima
Simulation Study : Gamblers Ruins
Simulation Study: Probability of Gambler Ruin

\end{frame}
%=====================================================================================================================%
\begin{frame}

Section 1: Basic R commands and Functions
Installing R on your computer
R can be easily downloaded from the Comprehenive R Archive Network (CRAN) website.
\end{frame}
%=====================================================================================================================%
\begin{frame}

Editing your Data

x=c(0 ,5)     	      # create a vector x
data.entry(x)  	   # edit the values using spreadsheet interface.
x  	                     # print to screen
x=edit(x)	          # the 'edit' function to call the script editor
x  	                     # print to screen

\end{frame}

%=====================================================================================================================%
\begin{frame}
Manipulating Characters
> nchar("oscar")
[1] 5

Objects
During an R session, objects are created and stored by name. The command "ls()" displays all currently-stored objects (workspace). Objects can be removed using the "rm()" function.

\end{frame}
%=====================================================================================================================%
\begin{frame}

ls()
rm(x, a, temp, wt.males)
rm(list=ls())								#removes all of the objects in the workspace.


At the end of each R session, you are prompted to save your workspace. If you click Yes, all objects are written to the ".RData" file. 
When R is re-started, it reloads the workspace from this file and the command history stored in ".Rhistory" is also reloaded.

%==================================================================================================================%


Section 3 : Data Structures
Vectors, Arrays and Matrices

Lists

Frames

Indexing

Subsetting

Section 4 : Regression models

Simple Linear Regression


Summary of SLR


SLR analysis involves

Creating a scatterplot to determine the nature of the relationship between x and y

If the relationship is linear, measuring the strength of the relationship using the correlation coe cient

Fitting the best model by estimating parameter values from data

There are always lots of di erent possible models to describe a given data set

Using diagnostic plots of the residuals to check the adequacy of the fitted model. Must check for non-constant variance and non-normal errors.

If the relationship is non-linear, t e.g. polynomial, exponential, non-linear model and use predict to generate the fitted curve for plotting



lm (y ~ x)
Multiple Linear Regression
variable selection procedures

Backward Selection

Forward Selection

Stepwise Selection
Non Linear Regression
Quadratic Regression





Histogram: sample code
x <- rnorm(1000) 
hx <- hist(x, breaks=100, plot=FALSE) 
plot(hx, col=ifelse(abs(hx$breaks) < 1.669, 4, 2)) 

# What is cool is that "col" is supplied a vector.

Data Management

Creating New Variables
Operators
Built-in Functions
Control Structures
User-defined Functions
Sorting Data
Merging Data
Aggregating Data
Aggregating Data
Reshaping Data
Subsetting Data
Data Type Conversion
Vector Functions

range(x) 	    # range
sum(x) 		  # sum
min(x)	        # minimum
max(x)	       # maximum
diff(x, lag=2)   # lagged differences, with lag indicating which lag to use



 


 
Section 8: Hypothesis Tests

The Correlation Test

cor.test(X,Y)


The Chi Square Test





The Shapiro Wilk Test

The Shapiro Wilk Test is another test for distribution test





The Chi Square Test for Independence





The PropTest





The F Test





The Kolmogorov Smirnov Test

X = rnorm(5,1,5)
Y = rexp(5)
ks.test(X,Y)


The Anderson Darling Test






The t-test







The ANOVA F-Test

 
Section 9: Graphics


Graphics Parameters


Section 10: Simulation

Simulation Study : Random Walks



P = 0.5 					#probability of a move to the right
Q = 1-P; S=Q/P;

Posn =0;N= 200;Trkr=numeric(N);Orgn=0; 

for (i in 1:N)
	{
	Trkr[i]=Posn
	if (P < runif(1)) Posn=Posn+1 else Posn=Posn-1
	if(Posn==0) Orgn=c(Orgn,i)
	}


plot(Trkr,type="o")
abline(h=0, col="red")
diff(Orgn)
Rogn = sort(diff(Orgn))
length(Rogn)
summary(as.factor(Rogn))
summary(as.factor(Rogn))[[1:10]]

Simulation Study: Distribution of pairwise maxima and minima



n=20
X<-rnorm(n) ; Y = rnorm(n); 
Mn =numeric(n) ;Mx = numeric(n);
for( i in 1:n)
{
W[i]=min(X[i],Y[i])
Z[i]=max(X[i],Y[i])
}

cbind(X,Y,W,Z)





Simulation Study: estimating a quantile from a probability distributions



N = 2000 #number of Loops
n = 200 #size of each sample

Qvec = numeric(N)
Q = 0.975

for (i in 1:N){
X = rnorm(n)
Qtl = quantile (X,Q)
Qvec[i] = Qtl
}

Qvec

mean(Qvec)
# Alternative method

# Qtls[i] =quantile(rnorm(n),Q)

