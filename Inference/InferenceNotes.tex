%-------------------------------------------------------------------------------------------------------------%
%{P-values}
Lets simulate 100 rolls of a die and sum up the numbers.\\
\begin{itemize}
\item The command \color{blue}\texttt{runif(100,1,7)}\color{black} generates 100 random numbers uniformly distributed between 1 and 7.
\item The command \color{blue}\texttt{floor()}\color{black} discretizes those values.
\item The command \color{blue}\texttt{sum()}\color{black} computes the sum of those 100 values.
\end{itemize}


\rule{12.1cm}{0.25mm}
\texttt{R} Code:
\begin{verbatim}
>
> x<-sum(floor(runif(100,1,7)))
> x
[1] 357
>
\end{verbatim}


%

%--------------------------------------------------------------------------------------------------------%

%--------------------------------------------------------------------------------------------------------%

%{P-values}
We could have used some other approachs in implementing this.\\
\begin{itemize}
\item The command \color{blue}\texttt{runif(100,0,6)}\color{black} generates 100 random numbers uniformly distributed between 0 and 6.
\item The command \color{blue}\texttt{ceiling()}\color{black} discretizes those values.
\item The command \color{blue}\texttt{sum()}\color{black} computes the sum of those 100 values.
\end{itemize}


\rule{12.1cm}{0.25mm}
\texttt{R} Code:
\begin{verbatim}
>
> x<-sum(ceiling(runif(100,0,6)))
> x
[1] 372
>
\end{verbatim}
%


%-----------------------------------------------
\subsection{Statistical Inference}
\begin{itemize}
\item $R$ commands for statistical inference procedures \item
t.test() - testing procedure for means.
\begin{itemize}
\item One sample \item Two sample \item Paired
\end{itemize}
\item prop.test() - testing procedure for proportions.
\begin{itemize}
\item One sample \item Two sample
\end{itemize}
\item var.test() - testing procedure for variances.
\end{itemize}
%




%-----------------------------------------------
\subsection{Single sample inference}

If we have a single sample we might want to answer several
questions:
\begin{itemize}
\item What is the mean value? \item Is the mean value
significantly different from current theory? (Hypothesis test)
\item What is the level of uncertainty associated with our
estimate of the mean value? (Confidence interval)
\end{itemize}
To ensure that our analysis is correct we need to check for
outliers in the data (i.e. boxplots) and we also need to check
whether the data are normally distributed or not.
%

%-----------------------------------------------
\subsection{Checking normality}


Graphical methods are often used to check that the data being
analysed are normally distributed. We can use
\begin{itemize}
\item Histogram - check for symmetry \item Boxplot - symmetry and
outliers \item Normal probability (Q-Q) plot

\item Other procedures
\begin{itemize}
\item Kolmogorov-Smirnov test (ks.test())\item Shapiro Wilk test (shapiro.test()) \item
Grubb's test \item Anderson Darling test
\end{itemize}
\end{itemize}
We shall revert to these tests later.
%


%-----------------------------------------------------------------------------------------------------------------------%
\subsection{Hypothesis testing for a mean}

\begin{itemize}
\item (Last week : confidence interval for a mean) \item Revision:
For large samples ($n > 30$) and/or if the population standard
deviation ($\sigma$) is known, the usual test statistic is given
by: \[Z =\frac{\bar{X} - \mu}{SE(\bar{X})}\]

\item $S.E.(\bar{X}) = { \sigma \over \sqrt{n}} $ or ${s \over \sqrt{n}}$. \item For small samples, use the $t-$distribution
with $n-1$ degrees of freedom.
\item
Critical value from tables.
\item Compare test statistics and critical values.
\end{itemize}


%



\rule{12.1cm}{0.25mm}
\texttt{R} Code:
\begin{verbatim}

> x = c(3, 0, 5, 2, 5, 5, 5, 4, 4, 5)
>qqnorm(x)
>qqline(x)

\end{verbatim}

%


\end{document}
