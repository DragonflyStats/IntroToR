Creation
Recall that we create a data.frame using the function data.frame():
> DF = data.frame(x=c("b","b","b","a","a"),v=rnorm(5))
> DF
x v
1 b -0.1758447
2 b -1.1006405
3 b 1.2296532
4 a -1.7714559
5 a -0.2651259
A data.table is created in exactly the same way:
> DT = data.table(x=c("b","b","b","a","a"),v=rnorm(5))
> DT
x v
1: b 0.7693975
2: b 2.6219549
3: b 0.6689401
4: a 0.7741039
5: a 0.4980158
Observe that a data.table prints the row numbers with a colon so as to visually separate the row
number from the first column. We can easily convert existing data.frame objects to data.table.
> CARS = data.table(cars)
> head(CARS)
speed dist
1: 4 2
2: 4 10
3: 7 4
4: 7 22
5: 8 16
6: 9 10
1
We have just created two data.tables: DT and CARS. It is often useful to see a list of all
data.tables in memory:
> tables()
NAME NROW NCOL MB COLS KEY
[1,] CARS 50 2 1 speed,dist
[2,] DT 5 2 1 x,v
Total: 2MB
The MB column is useful to quickly assess memory use and to spot if any redundant tables can
be removed to free up memory. Just like data.frames, data.tables must fit inside RAM.
Some users regularly work with 20 or more tables in memory, rather like a database. The result
of tables() is itself a data.table, returned silently, so that tables() can be used in programs.
tables() is unrelated to the base function table().
To see the column types :
> sapply(DT,class)
x v
"character" "numeric"
You may have noticed the empty column KEY in the result of tables() above. This is the
subject of the next section.
1. Keys
Let’s start by considering data.frame, specifically rownames. We know that each row has exactly
one row name. However, a person (for example) has at least two names, a first name and a second
name. It’s useful to organise a telephone directory sorted by surname then first name.
In data.table, a key consists of one or more columns. These columns may be integer, factor or
numeric as well as character. Furthermore, the rows are sorted by the key. Therefore, a data.table
can have at most one key because it cannot be sorted in more than one way. We can think of a
key as like super-charged row names; i.e., mult-column and multi-type.
Uniqueness is not enforced; i.e., duplicate key values are allowed. Since the rows are sorted by
the key, any duplicates in the key will appear consecutively.
Let’s remind ourselves of our tables:
> tables()
NAME NROW NCOL MB COLS KEY
[1,] CARS 50 2 1 speed,dist
[2,] DT 5 2 1 x,v
Total: 2MB
> DT
x v
1: b 0.7693975
2: b 2.6219549
3: b 0.6689401
4: a 0.7741039
5: a 0.4980158
No keys have been set yet.
> DT[2,] # select row 2
2
x v
1: b 2.621955
> DT[x=="b",] # select rows where column x == "b"
x v
1: b 0.7693975
2: b 2.6219549
3: b 0.6689401
Aside: notice that we did not need to prefix x with DT$x. In data.table queries, we can use
column names as if they are variables directly.
But since there are no rownames, the following does not work:
> cat(try(DT["b",],silent=TRUE))
Error in ❵[.data.table❵(DT, "b", ) :
When i is a data.table (or character vector), x must be keyed (i.e. sorted, and, marked as sorted) so data.table knows which columns to join to and take advantage of x being sorted. Call setkey(x,...) first, see ?setkey.
The error message tells us we need to use setkey():
> setkey(DT,x)
> DT
x v
1: a 0.7741039
2: a 0.4980158
3: b 0.7693975
4: b 2.6219549
5: b 0.6689401
Notice that the rows in DT have now been re-ordered according to the values of x. The two
"a" rows have moved to the top. We can confirm that DT does indeed have a key using haskey(),
key(), attributes(), or just running tables().
> tables()
NAME NROW NCOL MB COLS KEY
[1,] CARS 50 2 1 speed,dist
[2,] DT 5 2 1 x,v x
Total: 2MB
Now that we are sure DT has a key, let’s try again:
> DT["b",]
x v
1: b 0.7693975
2: b 2.6219549
3: b 0.6689401
By default all the rows in the group are returned1
. The mult argument (short for multiple)
allows the first or last row of the group to be returned instead.
> DT["b",mult="first"]
x v
1: b 0.7693975
1
In contrast to a data.frame where only the first rowname is returned when the rownames contain duplicates.
3
> DT["b",mult="last"]
x v
1: b 0.6689401
Also, the comma is optional.
> DT["b"]
x v
1: b 0.7693975
2: b 2.6219549
3: b 0.6689401
Let’s now create a new data.frame. We will make it large enough to demonstrate the difference
between a vector scan and a binary search.
> grpsize = ceiling(1e7/26^2) # 10 million rows, 676 groups
[1] 14793
> tt=system.time( DF <- data.frame(
+ x=rep(LETTERS,each=26*grpsize),
+ y=rep(letters,each=grpsize),
+ v=runif(grpsize*26^2),
+ stringsAsFactors=FALSE)
+ )
user system elapsed
0.448 0.008 0.457
> head(DF,3)
x y v
1 A a 0.5310106
2 A a 0.1980941
3 A a 0.8835322
> tail(DF,3)
x y v
10000066 Z z 0.6231946
10000067 Z z 0.4410910
10000068 Z z 0.9604099
> dim(DF)
[1] 10000068 3
We might say that R has created a 3 column table and inserted 10,000,068 rows. It took 0.457
secs, so it inserted 21,881,986 rows per second. This is normal in base R. Notice that we set
stringsAsFactors=FALSE. This makes it a little faster for a fairer comparison, but feel free to
experiment.
Let’s extract an arbitrary group from DF:
> tt=system.time(ans1 <- DF[DF$x=="R" & DF$y=="h",]) # ✬vector scan✬
user system elapsed
0.528 0.016 0.544
> head(ans1,3)
4
x y v
6642058 R h 0.7437625
6642059 R h 0.5702467
6642060 R h 0.3618726
> dim(ans1)
[1] 14793 3
Now convert to a data.table and extract the same group:
> DT = as.data.table(DF) # but normally use fread() or data.table() directly, originally
> system.time(setkey(DT,x,y)) # one-off cost, usually
user system elapsed
0.052 0.000 0.052
> ss=system.time(ans2 <- DT[list("R","h")]) # binary search
user system elapsed
0.004 0.000 0.001
> head(ans2,3)
x y v
1: R h 0.7437625
2: R h 0.5702467
3: R h 0.3618726
> dim(ans2)
[1] 14793 3
> identical(ans1$v, ans2$v)
[1] TRUE
At 0.001 seconds, this was 544 times faster than 0.544 seconds, and produced precisely the
same result. If you are thinking that a few seconds is not much to save, it’s the relative speedup
that’s important. The vector scan is linear, but the binary search is O(log n). It scales. If a task
taking 10 hours is sped up by 100 times to 6 minutes, that is significant2
.
We can do vector scans in data.table, too. In other words we can use data.table badly.
> system.time(ans1 <- DT[x=="R" & y=="h",]) # works but is using data.table badly
user system elapsed
0.464 0.016 0.479
> system.time(ans2 <- DF[DF$x=="R" & DF$y=="h",]) # the data.frame way
user system elapsed
0.536 0.008 0.543
> mapply(identical,ans1,ans2)
x y v
TRUE TRUE TRUE
2We wonder how many people are deploying parallel techniques to code that is vector scanning
5
If the phone book analogy helped, the 544 times speedup should not be surprising. We use
the key to take advantage of the fact that the table is sorted and use binary search to find the
matching rows. We didn’t vector scan; we didn’t use ==.
When we used x=="R" we scanned the entire column x, testing each and every value to see if
it equalled ”R”. We did it again in the y column, testing for ”h”. Then & combined the two logical
results to create a single logical vector which was passed to the [ method, which in turn searched
it for TRUE and returned those rows. These were vectorized operations. They occurred internally
in R and were very fast, but they were scans. We did those scans because we wrote that R code.
When i is a list (and data.table is a list too), we say that we are joining. In this case, we
are joining DT to the 1 row, 2 column table returned by list("R","h"). Since we do this a lot,
there is an alias for list: .().
> identical( DT[list("R","h"),],
+ DT[.("R","h"),])
[1] TRUE
Both vector scanning and binary search are available in data.table, but one way of using
data.table is much better than the other.
The join syntax is a short, fast to write and easy to maintain. Passing a data.table into a
data.table subset is analogous to A[B] syntax in base R where A is a matrix and B is a 2-column
matrix3
. In fact, the A[B] syntax in base R inspired the data.table package. There are other
types of ordered joins and further arguments which are beyond the scope of this quick introduction.
The merge method of data.table is very similar to X[Y], but there are some differences. See
FAQ 1.12.
This first section has been about the first argument inside DT[...], namely i. The next section
is about the 2nd and 3rd arguments: j and by.
2. Fast grouping
The second argument to DT[...] is j and may consist of one or more expressions whose arguments
are (unquoted) column names, as if the column names were variables. Just as we saw earlier in i
as well.
> DT[,sum(v)]
[1] 4999637
When we supply a j expression and a ’by’ expression, the j expression is repeated for each ’by’
group.
> DT[,sum(v),by=x]
x V1
1: A 192213.2
2: B 192183.3
3: C 192601.7
4: D 192308.0
5: E 192428.5
6: F 192071.0
7: G 192403.8
8: H 192423.9
9: I 192024.5
10: J 192063.1
11: K 192340.2
3Subsetting a keyed data.table by a n-column data.table is consistent with subsetting a n-dimension array by
a n-column matrix in base R
6
12: L 192421.5
13: M 192470.2
14: N 192045.5
15: O 192166.7
16: P 192459.4
17: Q 192307.1
18: R 192288.1
19: S 192274.7
20: T 192380.5
21: U 192191.0
22: V 192170.7
23: W 192257.5
24: X 192401.6
25: Y 192429.4
26: Z 192312.4
x V1
The by in data.table is fast. Let’s compare it to tapply.
> ttt=system.time(tt <- tapply(DT$v,DT$x,sum)); ttt
user system elapsed
0.704 0.064 0.767
> sss=system.time(ss <- DT[,sum(v),by=x]); sss
user system elapsed
0.080 0.000 0.078
> head(tt)
A B C D E F
192213.2 192183.3 192601.7 192308.0 192428.5 192071.0
> head(ss)
x V1
1: A 192213.2
2: B 192183.3
3: C 192601.7
4: D 192308.0
5: E 192428.5
6: F 192071.0
> identical(as.vector(tt), ss$V1)
[1] TRUE
At 0.078 sec, this was 9 times faster than 0.767 sec, and produced precisely the same result.
Next, let’s group by two columns:
> ttt=system.time(tt <- tapply(DT$v,list(DT$x,DT$y),sum)); ttt
user system elapsed
1.076 0.080 1.155
> sss=system.time(ss <- DT[,sum(v),by="x,y"]); sss
user system elapsed
0.104 0.000 0.106
7
> tt[1:5,1:5]
a b c d e
A 7382.299 7424.815 7345.469 7347.148 7356.512
B 7360.890 7383.625 7348.990 7381.238 7430.159
C 7432.864 7433.346 7398.234 7429.309 7406.106
D 7387.108 7398.470 7390.907 7402.977 7393.608
E 7399.820 7435.018 7374.863 7396.102 7399.262
> head(ss)
x y V1
1: A a 7382.299
2: A b 7424.815
3: A c 7345.469
4: A d 7347.148
5: A e 7356.512
6: A f 7411.005
> identical(as.vector(t(tt)), ss$V1)
[1] TRUE
This was 10 times faster, and the syntax is a little simpler and easier to read.
3. Fast ordered joins
This is also known as last observation carried forward (LOCF) or a rolling join.
Recall that X[Y] is a join between data.table X and data.table Y. If Y has 2 columns, the
first column is matched to the first column of the key of X and the 2nd column to the 2nd. An
equi-join is performed by default, meaning that the values must be equal.
Instead of an equi-join, a rolling join is :
X[Y,roll=TRUE]
As before the first column of Y is matched to X where the values are equal. The last join column
in Y though, the 2nd one in this example, is treated specially. If no match is found, then the row
before is returned, provided the first column still matches.
Further controls are rolling forwards, backwards, nearest and limited staleness.
For examples type example(data.table) and follow the output at the prompt.
