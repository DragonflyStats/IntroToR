 
 
 
 Base packages
 Base packages are considered part of the R source code. These packages contain the basic functions that allow R to work, and include many commonly used datasets and standard statistical and graphical functions. They should be automatically available in any R installation.
 
 
 To see which packages are installed on your computer, use the library() command, with no arguments.
 
 Contributed packages
 Hundreds of contributed packages covering a wide array of modern statistical methods are available from the Comprehensive R Archive Network (CRAN; http: //cran.r-project.org).
 
 To access all of the functions and data sets in a particular package, it must be loaded into the workspace. 
 For example, to load the lattice package:
 > library(lattice)
 
 To remove a package from a session, use the detach command
 >detach(package:lattice)
 
 Packages must be reloaded at the start of each new session.
 
 
 library(MASS)							  #loads the MASS package
 
 install.packages("evir")			#install the "evir" package (select the Irish CRAN Mirror)
 library(evir)
 
 
\section{Packages}

Type \texttt{library()} onto the command line to find out what packages were installed on your computer
when you put in \texttt{R}. Now type \texttt{search()} to get a list of the packages from that list that are
already loaded in (installed packages are not necessarily loaded in). 

%------------------%

%------------------------------------------------------------%

\subsection{Packages}

\begin{verbatim}
install.package("packagename")
download.package("packagename")
\end{verbatim}
%---------------------------%
\end{document}
