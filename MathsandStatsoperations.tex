\documentclass[a4paper,12pt]{article}
%%%%%%%%%%%%%%%%%%%%%%%%%%%%%%%%%%%%%%%%%%%%%%%%%%%%%%%%%%%%%%%%%%%%%%%%%%%%%%%%%%%%%%%%%%%%%%%%%%%%%%%%%%%%%%%%%%%%%%%%%%%%%%%%%%%%%%%%%%%%%%%%%%%%%%%%%%%%%%%%%%%%%%%%%%%%%%%%%%%%%%%%%%%%%%%%%%%%%%%%%%%%%%%%%%%%%%%%%%%%%%%%%%%%%%%%%%%%%%%%%%%%%%%%%%%%
\usepackage{eurosym}
\usepackage{vmargin}
\usepackage{amsmath}
\usepackage{graphics}
\usepackage{epsfig}
\usepackage{subfigure}
\usepackage{fancyhdr}
%\usepackage{listings}
\usepackage{framed}
\usepackage{graphicx}

\setcounter{MaxMatrixCols}{10}
%TCIDATA{OutputFilter=LATEX.DLL}
%TCIDATA{Version=5.00.0.2570}
%TCIDATA{<META NAME="SaveForMode" CONTENT="1">}
%TCIDATA{LastRevised=Wednesday, February 23, 2011 13:24:34}
%TCIDATA{<META NAME="GraphicsSave" CONTENT="32">}
%TCIDATA{Language=American English}

\pagestyle{fancy}
\setmarginsrb{20mm}{0mm}{20mm}{25mm}{12mm}{11mm}{0mm}{11mm}
\lhead{Dublin \texttt{R}} \rhead{10 April 2013}
\chead{Introduction to \texttt{R} (Module A)}
%\input{tcilatex}


% http://www.norusis.com/pdf/SPC_v13.pdf
\begin{document}

\tableofcontents
\section{Mathematical and Statistical Commands}

\subsection{Useful Statistical Commands}
\begin{itemize}
	\item \texttt{mean()} mean of a data set
	\item \texttt{median()} median of a data set
	\item \texttt{length()} Sample Size
	\item \texttt{IQR()} Inter-Quartile Range of a sample
	\item \texttt{var()} Variance of a sample
	\item \texttt{sd()} Standard Deviation  of a sample
	\item \texttt{range()} Range of a data set
	\item \texttt{fivenum()} Tukey's five number summary
\end{itemize}

\subsection{useful operators}

\begin{itemize}
	\item Factorials
	$n! = n \times n-1 \times \ldots \times 2 \times 1 $
	\item Binomial Coefficients
	\[ { n \choose k }  = \frac{n!}{(n-k)! \times k!}\]
\end{itemize}
The \texttt{R} commands are \texttt{factorial()} and \texttt{choose()} respectively.
%--------------------%
\subsection{Managing Precision}

\begin{itemize}
	\item \texttt{floor()} Floor function of x, $\lfloor x \rfloor$.
	\item \texttt{ceiling()} Ceiling function of x, $\lceil x \rceil$.
	\item \texttt{round()} Rounding a number to a specified number of decimal places.
\end{itemize}
%--------------------------------------------%
\subsection{The Birthday function}
The R command pbirthday() computes the probability of a coincidence of a number of randomly chosen people sharing a birthday, given that there are n people to choose from.
Suppose there are four people in a room. The probability of two of them sharing a birthday is computed as about 1.6 \%
\begin{verbatim}
> pbirthday(4)
[1] 0.01635591
\end{verbatim}

How many people do you need for a greater than 50\% chance of a shared birthday? (choose from 23,43,63,83)?

%--------------------%
\subsection{Set Theory Operations}
\begin{itemize}
	\item \texttt{union()} union of sets A and B
	\item \texttt{intersect()} intersection of sets A and B
	\item \texttt{setdiff()} set difference A-B (order is important)
\end{itemize}

\begin{framed}
	\begin{verbatim}
	x = 5:10
	y = 8:12
	union(x,y)
	intersect(x,y)
	setdiff(x,y)
	setdiff(y,x)
	\end{verbatim}
\end{framed}


%----------------------------------------------------------------------------%
\newpage
%---------------------------%
\section{Mathematical and Statistical Commands}

\subsection{Useful Statistical Commands}
\begin{itemize}
\item \texttt{mean()} mean of a data set
\item \texttt{median()} median of a data set
\item \texttt{length()} Sample Size
\item \texttt{IQR()} Inter-Quartile Range
\item \texttt{var()} variance
\item \texttt{sd()} Standard Deviation
\item \texttt{range()} Range of a data set
\end{itemize}

\subsection{Useful Operators}

Factorials
$n! = n \times n-1 \times \ldots \times 2 \times 1 $
Binomial Coefficients
\[ { n \choose k }  = \frac{n!}{(n-k)! \times k!}\]
%--------------------%
\subsection{Managing Precision}

\begin{itemize}
\item \texttt{floor()} Floor function of x, $\lfloor x \rfloor$.
\item \texttt{ceiling()} Ceiling function of x, $\lceil x \rceil$.
\item \texttt{round()} Rounding a number to a specified number of decimal places.
\end{itemize}
%--------------------------------------------%
\subsection{The Birthday function}
The R command pbirthday() computes the probability of a coincidence of a number of randomly chosen people sharing a birthday, given that there are n people to choose from.
Suppose there are four people in a room. The probability of two of them sharing a birthday is computed as about 1.6 \%
\begin{verbatim}
> pbirthday(4)
[1] 0.01635591
\end{verbatim}

How many people do you need for a greater than 50\% chance of a shared birthday? (choose from 23,43,63,83)?
%---------------------------------------------------------%
\subsection{Mathematical Precision Functions}
Three commonly used mathematical precision functions are:
\begin{itemize}
	\item Absolute Value Function $| x |$ - distance on the number line from zero.
	\item Ceiling Function $\lceil x \rceil$ - rounds a value up to the nearest integer.
	\item Floor Function  $\lfloor x \rfloor $ - rounds a value down to the nearest integer.
\end{itemize}
\begin{framed}
	\begin{verbatim}
	pi
	floor(pi)
	ceiling(pi)
	\end{verbatim}
\end{framed}
\begin{verbatim}
> pi
[1] 3.141593
>
> floor(pi)
[1] 3
>
> ceiling(pi)
[1] 4
>
\end{verbatim}
We can also round numbers to a specified number of decimal places, using the \texttt{round()} command.
\begin{framed}
	\begin{verbatim}
	round(pi,3)
	round(pi,2)
	\end{verbatim}
\end{framed}
\begin{verbatim}
> round(pi,3)
[1] 3.142
> round(pi,2)
[1] 3.14
\end{verbatim}

\section{Sampling}
#########################################
# 
# Sampling
# 
# The R command we use to perform sampling is sample().
# 
# All elements in either X or Y




> X=c(4,5)
>
> sample(X,2)
[1] 4 5
>
> sample(X,1);sample(X,1);sample(X,1);
[1] 4
[1] 5
[1] 5

When x is a single value, the function sample() behaves differently.

> Y=c(4)
>
> sample(Y,1)
[1] 2
> 
> sample(Y,2)
[1] 3 1


\begin{itemize}
	\item Sampling With Replacement
	\item Sampling Without Replacement
\end{itemize}
%-----------------------------------------------------%
Generate a quick pick : pick 6 numbers from 1 to 42. ( Same number cant be selected more than once)

Generate five values from a die ( Same number can be selected more than once).

\begin{verbatim}
> Lotto = 1:42
> Dice = 1:6
> 
> sample(Lotto,6)
[1] 38 25 34 30 22 29
> 
> sample(Dice,5,replace = TRUE)
[1] 4 3 2 3 3
>
\end{verbatim}
%---------------------------------------------------------------------------%
\end{document}



%---------------------------------------------------------%

%--------------------%
\subsection{Set Theory Operations}
\begin{itemize}
\item \texttt{union()} union of sets A and B
\item \texttt{intersect()} intersection of sets A and B
\item \texttt{setdiff()} set difference A-B (order is important)
\end{itemize}

\begin{framed}
\begin{verbatim}
x = 5:10
y = 8:12
union(x,y)
intersect(x,y)
setdiff(x,y)
setdiff(y,x)
\end{verbatim}
\end{framed}
# R Class : Set Theory and Sampling

%=============================================================================%
%# Sampling
%# - Sampling With Replacement
%# - Sampling Without Replacement

%=============================================================================%
\begin{frame}[fragile] 
	\frametitle{Set Theory with \texttt{R} }
	
	\begin{itemize}
		\item Union
		\item Intersection
		\item Set Diffference
	\end{itemize}
	\begin{framed}
		\begin{verbatim}
		X = 5:10
		Y = 8:12
		\end{verbatim}
	\end{framed}
\end{frame}
%=============================================================================%
\begin{frame}[fragile] 
	\frametitle{Set Theory with \texttt{R} }
	\begin{framed}
		\begin{verbatim}
		
		union(X,Y)
		# [1]  5  6  7  8  9 10 11 12
		intersect(X,Y)
		# [1]  8  9 10
		
		\end{verbatim}
	\end{framed}
\end{frame}
%=============================================================================%
\end{document}

\subsection{Other Mathematical Functions}
%-----------------------------------------------------------------------%
Complex numbers
x = -1 ;  sqrt(x)  ;  str(x) ; 	# variable is defined as numeric, not complex.
y = -1 +0i ;  sqrt(y)  ;  str(y) ;    	#variable is defined as complex .
%-----------------------------------------------------------------------%
Trigonometric  Functions
pi				#returns the value of pi to six decimal places
sin(3.5*pi)			# correct answer is -1
cos(3.5*pi)			# correct answer is zero

%-----------------------------------------------------------------------%

%============================================================================================================== %
\subsection{Generating Random Numbers}
R is very useful for performing simulations.

\begin{framed}
\begin{verbatim}
#generate a random number between 0 and 1
runif(1)						
#generate four random numbers between 0 and 6				 
runif(4,min=0,max=6) 			  
\end{verbatim}
\end{framed}
Random numbers can be discretized using the "floor()" or "Ceiling()" functions. Suppose we wish to simulate four throws of a dice.

X = ceiling (runif(4,min=0,max=6))
Y = floor (runif(4,min=1,max=7))
X+Y

%================================================================================================================= %
\subsection{Section 2: Basic Mathematical operations}
Trigonometric and power functions

Integration

integrate(sin, lower =0, upper = 3)
integrate(dnorm, -1.96, 1.96)					 # standard normal distribution
integrate(dnorm, 0, Inf)							   # standard normal distribution

Complex numbers 

Matrices and Linear Algebra
Factorials and permutations
The Choose Function

63=654321= 20



> factorial(6)
[1] 720
> choose(6,3)
[1] 20
> 

\subsection{Basic Calculations}

We will briefly look at how R accomplished basic calculations.


x*y			# multiplication
x/z			# division

x^2			# powers
sqrt(x)		# square root

exp(z)		 # exponentials   
log(y)		 # logarithms

pi             # returns the value of pi to six decimal places


Complex numbers , Trigonometric  Functions and Binomial Coefficients


Binomial coefficients are computed using the choose() command.




J = -1 ;  sqrt(J)  ;  str(J) ;      # variable is defined as numeric, not complex.
K = -1 +0i ;  sqrt(K)  ;  str(K) ;  # variable is defined as complex .


sin(3.5*pi)             # correct answer is -1
cos(3.5*pi)             # correct answer is zero

choose(6,2)             # From 6 how many ways of choosing items.


%=========================================================================================================== %

\subsection{Truncation and discretization}

The functions "floor" and "ceiling" can be used to discretize outcomes. In this instance we should use "ceiling".

ceiling(X)
floor(X)
round(X,2)

X=ceiling(X)



The expected value is 3.5.

The variance from first principles we can calculate the variance using




\end{document}
