\documentclass[a4paper,12pt]{article}
%%%%%%%%%%%%%%%%%%%%%%%%%%%%%%%%%%%%%%%%%%%%%%%%%%%%%%%%%%%%%%%%%%%%%%%%%%%%%%%%%%%%%%%%%%%%%%%%%%%%%%%%%%%%%%%%%%%%%%%%%%%%%%%%%%%%%%%%%%%%%%%%%%%%%%%%%%%%%%%%%%%%%%%%%%%%%%%%%%%%%%%%%%%%%%%%%%%%%%%%%%%%%%%%%%%%%%%%%%%%%%%%%%%%%%%%%%%%%%%%%%%%%%%%%%%%
\usepackage{eurosym}
\usepackage{vmargin}
\usepackage{amsmath}
\usepackage{graphics}
\usepackage{epsfig}
\usepackage{subfigure}
\usepackage{fancyhdr}
%\usepackage{listings}
\usepackage{framed}
\usepackage{graphicx}

\setcounter{MaxMatrixCols}{10}
%TCIDATA{OutputFilter=LATEX.DLL}
%TCIDATA{Version=5.00.0.2570}
%TCIDATA{<META NAME="SaveForMode" CONTENT="1">}
%TCIDATA{LastRevised=Wednesday, February 23, 2011 13:24:34}
%TCIDATA{<META NAME="GraphicsSave" CONTENT="32">}
%TCIDATA{Language=American English}

\pagestyle{fancy}
\setmarginsrb{20mm}{0mm}{20mm}{25mm}{12mm}{11mm}{0mm}{11mm}
\lhead{Dublin \texttt{R}} \rhead{10 April 2013}
\chead{Introduction to \texttt{R} (Module A)}
%\input{tcilatex}


% http://www.norusis.com/pdf/SPC_v13.pdf
\begin{document}

\tableofcontents
\section{Mathematical and Statistical Commands}
%------------------------------------------------------%
1.8 Basic Maths Operations
The most commonly used mathematical operators are all supported by R. Here are a few
examples:
5 + 3 * 5 # Note the order of operations.
log (10) # Natural logarithm with base e=2.718282
log(8,2) # Log to the base 2
4^2 # 4 raised to the second power
7/2 # Division
factorial(4) #Factorial of Four
sqrt (25) # Square root
abs (3-7) # Absolute value of 3-7
pi # The mysterious number \\\
exp(2) # exponential function

%------------------------------------------------------%
R can be used for many mathematical operations, including
\item Set Theory
\item Trigonometry
\item Complex Numbers
\item Binomial Coecients
We will not go into any of these in great detail today.
%---------------------------------------------------------------------------------------%

\subsection{Basic Mathematical Calculations}
Basic operations
addition  +      
subtraction    	-
division   /     
multiplication   $\ast$
power     


Mathematical functions
\begin{itemize}
\item abs()       - Absolute value 
\item exp()	   - The exponential
\item log(,b)     - logarithm to the base "b". The default setting is the natural log.
\end{itemize}
\subsubsection*{trigonometrics functions}

pi            -  to 6 decimal places.

tan() ,cos(), sin()   - Common trigonometric functions

 
 Precision
 floor() 	  The floor function 		x
 ceiling()    The ceiling function       x

\begin{itemize} 
\item round()      Round to the nearest integer
\item round( ,2)  Round to two decimal places
\end{itemize} 
 
\begin{itemize}
\item Exercise:  write $\pi$   with 4 decimal places.  
\end{itemize}

 
%=====================================================================================================================%
\begin{framed}
\begin{verbatim}
	Complex numbers
	x = -1 ;  sqrt(x)  ;  str(x) ; 	# variable is defined as numeric, not complex.
	y = -1 +0i ;  sqrt(y)  ;  str(y) ;    	#variable is defined as complex .
	Trigonometric  Functions
	pi				#returns the value of pi to six decimal places
	sin(3.5*pi)			# correct answer is -1
	cos(3.5*pi)			# correct answer is zero
\end{verbatim}
\end{framed}
%=====================================================================================================================%

\subsection{Basic Calculations}
\begin{verbatim}
We will briefly look at how R accomplished basic calculations.


x*y			# multiplication
x/z			# division

x^2			# powers
sqrt(x)		# square root

exp(z)		 # exponentials   
log(y)		 # logarithms

pi             # returns the value of pi to six decimal places

\end{verbatim}

Complex numbers , Trigonometric  Functions and Binomial Coefficients


Binomial coefficients are computed using the choose() command.



\begin{framed}
\begin{verbatim}
J = -1 ;  sqrt(J)  ;  str(J) ;      # variable is defined as numeric, not complex.
K = -1 +0i ;  sqrt(K)  ;  str(K) ;  # variable is defined as complex .


\end{verbatim}
\end{framed}
% sin(3.5*pi)             # correct answer is -1
% cos(3.5*pi)             # correct answer is zero
% choose(6,2)             # From 6 how many ways of choosing items.


%=========================================================================================================== %






\subsection{Useful Mathematical Operators}

\begin{itemize}
	\item Factorials
	$n! = n \times n-1 \times \ldots \times 2 \times 1 $
	\item Binomial Coefficients
	\[ { n \choose k }  = \frac{n!}{(n-k)! \times k!}\]
\end{itemize}
The \texttt{R} commands are \texttt{factorial()} and \texttt{choose()} respectively.

Matrices and Linear Algebra
Factorials and permutations
The Choose Function

\[ { 6 \choose 3} =\frac{654}{321}= 20\]


\begin{framed}
\begin{verbatim}
> factorial(6)
[1] 720
> choose(6,3)
[1] 20
> 
\end{verbatim}
\end{framed}

%--------------------%
\subsection{Mathematical Precision Functions}
Three commonly used mathematical precision functions are:
\begin{itemize}
	\item Absolute Value Function $| x |$ - distance on the number line from zero.
	\item Ceiling Function $\lceil x \rceil$ - rounds a value up to the nearest integer.
	\item Floor Function  $\lfloor x \rfloor $ - rounds a value down to the nearest integer.
\end{itemize}
\begin{itemize}
	\item \texttt{floor()} Floor function of x, $\lfloor x \rfloor$.
	\item \texttt{ceiling()} Ceiling function of x, $\lceil x \rceil$.
	\item \texttt{round()} Rounding a number to a specified number of decimal places.
\end{itemize}
%--------------------------------------------%

\begin{framed}
	\begin{verbatim}
	pi
	floor(pi)
	ceiling(pi)
	\end{verbatim}
\end{framed}
\begin{verbatim}
> pi
[1] 3.141593
>
> floor(pi)
[1] 3
>
> ceiling(pi)
[1] 4
>
\end{verbatim}
We can also round numbers to a specified number of decimal places, using the \texttt{round()} command.
\begin{framed}
	\begin{verbatim}
	round(pi,3)
	round(pi,2)
	\end{verbatim}
\end{framed}
\begin{verbatim}
> round(pi,3)
[1] 3.142
> round(pi,2)
[1] 3.14
\end{verbatim}


\subsection{Truncation and discretization}

The functions "floor" and "ceiling" can be used to discretize outcomes. In this instance we should use "ceiling".

ceiling(X)
floor(X)
round(X,2)

X=ceiling(X)



The expected value is 3.5.

The variance from first principles we can calculate the variance using

\subsection{Managing Precision}

\begin{itemize}
	\item \texttt{floor()} Floor function of x, $\lfloor x \rfloor$.
	\item \texttt{ceiling()} Ceiling function of x, $\lceil x \rceil$.
	\item \texttt{round()} Rounding a number to a specified number of decimal places.
\end{itemize}

%--------------------%
\subsection{Sequences}


\subsection{Sampling}

Types of Sampling
\begin{itemize}
	\item Sampling With Replacement
	\item Sampling Without Replacement
\end{itemize}

The R command we use to perform sampling is sample().

All elements in either X or Y

\begin{framed}
	\begin{verbatim}
	
	
	> X=c(4,5)
	>
	> sample(X,2)
	[1] 4 5
	>
	> sample(X,1);sample(X,1);sample(X,1);
	[1] 4
	[1] 5
	[1] 5
	
	\end{verbatim}
\end{framed}
When x is a single value, the function sample() behaves differently.

\begin{framed}
	\begin{verbatim}
	
	> Y=c(4)
	>
	> sample(Y,1)
	[1] 2
	> 
	> sample(Y,2)
	[1] 3 1
	\end{verbatim}
\end{framed}


%-----------------------------------------------------%
Generate a quick pick : pick 6 numbers from 1 to 42. ( Same number cant be selected more than once)

Generate five values from a die ( Same number can be selected more than once).

\begin{verbatim}
> Lotto = 1:42
> Dice = 1:6
> 
> sample(Lotto,6)
[1] 38 25 34 30 22 29
> 
> sample(Dice,5,replace = TRUE)
[1] 4 3 2 3 3
>
\end{verbatim}
%---------------------------------------------------------------------------%




%---------------------------------------------------------%

%--------------------%

#########################################
# 
# Sampling
# 
# The R command we use to perform sampling is sample().
# 
# All elements in either X or Y




> X=c(4,5)
>
> sample(X,2)
[1] 4 5
>
> sample(X,1);sample(X,1);sample(X,1);
[1] 4
[1] 5
[1] 5

When x is a single value, the function sample() behaves differently.

> Y=c(4)
>
> sample(Y,1)
[1] 2
> 
> sample(Y,2)
[1] 3 1


\begin{itemize}
	\item Sampling With Replacement
	\item Sampling Without Replacement
\end{itemize}
%-----------------------------------------------------%
Generate a quick pick : pick 6 numbers from 1 to 42. ( Same number cant be selected more than once)

Generate five values from a die ( Same number can be selected more than once).

\begin{verbatim}
> Lotto = 1:42
> Dice = 1:6
> 
> sample(Lotto,6)
[1] 38 25 34 30 22 29
> 
> sample(Dice,5,replace = TRUE)
[1] 4 3 2 3 3
>
\end{verbatim}
%---------------------------------------------------------------------------%
\end{document}

\subsection{Useful Statistical Commands}
\begin{itemize}
	\item \texttt{mean()} mean of a data set
	\item \texttt{median()} median of a data set
	\item \texttt{length()} Sample Size
	\item \texttt{IQR()} Inter-Quartile Range of a sample
	\item \texttt{var()} Variance of a sample
	\item \texttt{sd()} Standard Deviation  of a sample
	\item \texttt{range()} Range of a data set
	\item \texttt{fivenum()} Tukey's five number summary
\end{itemize}

\subsection{Set Theory Operations}
\begin{itemize}
	\item \texttt{union()} union of sets A and B
	\item \texttt{intersect()} intersection of sets A and B
	\item \texttt{setdiff()} set difference A-B (order is important)
\end{itemize}

\begin{framed}
	\begin{verbatim}
	x = 5:10
	y = 8:12
	union(x,y)
	intersect(x,y)
	setdiff(x,y)
	setdiff(y,x)
	\end{verbatim}
\end{framed}
%=============================================================================%


% R Class : Set Theory and Sampling

%# Sampling
%# - Sampling With Replacement
%# - Sampling Without Replacement

%=============================================================================%
\subsection{Set Theory with \texttt{R} }
	
	\begin{itemize}
		\item Union
		\item Intersection
		\item Set Diffference
	\end{itemize}
	\begin{framed}
		\begin{verbatim}
		X = 5:10
		Y = 8:12
		\end{verbatim}
	\end{framed}

	\begin{framed}
		\begin{verbatim}
		
		union(X,Y)
		# [1]  5  6  7  8  9 10 11 12
		intersect(X,Y)
		# [1]  8  9 10
		
		\end{verbatim}
	\end{framed}
%=============================================================================%


# R Class : Set Theory and Sampling

%=============================================================================%
%# Sampling
%# - Sampling With Replacement
%# - Sampling Without Replacement

%=============================================================================%
\begin{frame}[fragile] 
	\frametitle{Set Theory with \texttt{R} }
	
	\begin{itemize}
		\item Union
		\item Intersection
		\item Set Diffference
	\end{itemize}
	\begin{framed}
		\begin{verbatim}
		X = 5:10
		Y = 8:12
		\end{verbatim}
	\end{framed}
\end{frame}
%=============================================================================%
\begin{frame}[fragile] 
	\frametitle{Set Theory with \texttt{R} }
	\begin{framed}
		\begin{verbatim}
		
		union(X,Y)
		# [1]  5  6  7  8  9 10 11 12
		intersect(X,Y)
		# [1]  8  9 10
		
		\end{verbatim}
	\end{framed}
\end{frame}
%=============================================================================%
\end{document}

\subsection{Set Theory Operations}
\begin{itemize}
	\item \texttt{union()} union of sets A and B
	\item \texttt{intersect()} intersection of sets A and B
	\item \texttt{setdiff()} set difference A-B (order is important)
\end{itemize}

\begin{framed}
	\begin{verbatim}
	x = 5:10
	y = 8:12
	union(x,y)
	intersect(x,y)
	setdiff(x,y)
	setdiff(y,x)
	\end{verbatim}
\end{framed}
%---------------------------------------------------------%
\subsection{The Birthday function}
The R command \texttt{pbirthday()} computes the probability of a coincidence of a number of randomly chosen people sharing a birthday, given that there are n people to choose from.
Suppose there are four people in a room. The probability of two of them sharing a birthday is computed as about 1.6 \%
\begin{verbatim}
> pbirthday(4)
[1] 0.01635591
\end{verbatim}

How many people do you need for a greater than 50\% chance of a shared birthday? (choose from 23,43,63,83)?
%---------------------------%



\subsection{Other Mathematical Functions}
%-----------------------------------------------------------------------%
Complex numbers
\begin{framed}
	\begin{verbatim}
	x = -1 ;  sqrt(x)  ;  str(x) ; 	# variable is defined as numeric, not complex.
y = -1 +0i ;  sqrt(y)  ;  str(y) ;    	#variable is defined as complex .

\end{verbatim}
\end{framed}
%-----------------------------------------------------------------------%
Trigonometric  Functions

\begin{framed}
\begin{verbatim}
pi				#returns the value of pi to six decimal places
sin(3.5*pi)			# correct answer is -1
cos(3.5*pi)			# correct answer is zero

\end{verbatim}
\end{framed}
%-----------------------------------------------------------------------%

%============================================================================================================== %
\subsection{Generating Random Numbers}
R is very useful for performing simulations.

\begin{framed}
\begin{verbatim}
#generate a random number between 0 and 1
runif(1)						
#generate four random numbers between 0 and 6				 
runif(4,min=0,max=6) 			  
\end{verbatim}
\end{framed}
Random numbers can be discretized using the "floor()" or "Ceiling()" functions. Suppose we wish to simulate four throws of a dice.

X = ceiling (runif(4,min=0,max=6))
Y = floor (runif(4,min=1,max=7))
X+Y

%================================================================================================================= %
\subsection{Section 2: Basic Mathematical operations}
Trigonometric and power functions

Integration

\begin{framed}
\begin{verbatim}
integrate(sin, lower =0, upper = 3)
integrate(dnorm, -1.96, 1.96)					 # standard normal distribution
integrate(dnorm, 0, Inf)							   # standard normal distribution
\end{verbatim}
\end{framed}
Complex numbers 

		
		\[\mbox{Computing with \texttt{R}}\]
		\LARGE
		\[\mbox{Built-in Constants}\]
		
		
		\texttt{R} has a small number of built-in constants.
		
		
		\begin{description}
			\item[\texttt{LETTERS}]: the 26 upper-case letters of the Roman alphabet;
			\bigskip
			\item[\texttt{letters}]: the 26 lower-case letters of the Roman alphabet;
			\bigskip
			\item[\texttt{month.abb}]: the three-letter abbreviations for the English month names;
			\bigskip
			\item[\texttt{month.name}]: the English names for the months of the year;
			\bigskip
			\item[\texttt{pi}]: the ratio of the circumference of a circle to its diameter.
		\end{description}
		
\end{document}
