

%==============================================%
\begin{frame}[fragile]
\frametitle{Lattice Graphs}
\begin{itemize}
\item The lattice package, written by Deepayan Sarkar, attempts to improve on base R graphics by providing better defaults and the ability to easily display multivariate relationships.
\item In particular, the package supports the creation of trellis graphs - graphs that display a variable or the relationship between variables, conditioned on one or more other variables.
\end{itemize}
\end{frame}
%==============================================%
\begin{frame}[fragile]
\begin{framed}
\begin{verbatim}
The typical format is
\begin{verbatim}
graph_type(formula, data=)
\end{verbatim}
where graph\_type is selected from the listed below. formula specifies the variable(s) to display and any conditioning variables . For example ~x|A means display numeric variable x for each level of factor A. y~x | A*B means display the relationship between numeric variables y and x separately for every combination of factor A and B levels. ~x means display numeric variable x alone.


graph_type	description	formula examples
barchart	bar chart	x~A or A~x
bwplot	boxplot	x~A or A~x
cloud	3D scatterplot	z~x*y|A
contourplot	3D contour plot	z~x*y
densityplot	kernal density plot	~x|A*B
dotplot	dotplot	~x|A
histogram	histogram	~x
levelplot	3D level plot	z~y*x
parallel	parallel coordinates plot	data frame
splom	scatterplot matrix	data frame
stripplot	strip plots	A~x or x~A
xyplot	scatterplot	y~x|A
wireframe	3D wireframe graph	z~y*x


Here are some examples. They use the car data (mileage, weight, number of gears, number of cylinders, etc.) from the mtcars data frame.

\end{frame}
%==============================================%
\begin{frame}[fragile]
\begin{framed}
\begin{verbatim}
# Lattice Examples 
library(lattice) 
attach(mtcars)
\end{verbatim}
\end{framed}
\end{frame}
%==============================================%
\begin{frame}[fragile]
\begin{framed}
\begin{verbatim}
# create factors with value labels 
gear.f<-factor(gear,levels=c(3,4,5),
  	labels=c("3gears","4gears","5gears")) 
cyl.f <-factor(cyl,levels=c(4,6,8),
   labels=c("4cyl","6cyl","8cyl")) 
\end{verbatim}
\end{framed}
\end{frame}
%==============================================%
\begin{frame}[fragile]
\begin{framed}
\begin{verbatim}
# kernel density plot 
densityplot(~mpg, 
  	main="Density Plot", 
  	xlab="Miles per Gallon")
\end{verbatim}
\end{framed}
\end{frame}
%==============================================%
\begin{frame}[fragile]
\begin{framed}
\begin{verbatim}
# kernel density plots by factor level 
densityplot(~mpg|cyl.f, 
  	main="Density Plot by Number of Cylinders",
   xlab="Miles per Gallon")
\end{verbatim}
\end{framed}
\end{frame}
%==============================================%
\begin{frame}[fragile]
\begin{framed}
\begin{verbatim}
# kernel density plots by factor level (alternate layout) 
densityplot(~mpg|cyl.f, 
  	main="Density Plot by Numer of Cylinders",
   xlab="Miles per Gallon", 
   layout=c(1,3))

\end{verbatim}
\end{framed}
\end{frame}
%==============================================%
\begin{frame}[fragile]
\begin{framed}
\begin{verbatim}
# boxplots for each combination of two factors 
bwplot(cyl.f~mpg|gear.f,
  	ylab="Cylinders", xlab="Miles per Gallon", 
   main="Mileage by Cylinders and Gears", 
   layout=(c(1,3))
\end{verbatim}
\end{framed}
\end{frame}
%==============================================%
\begin{frame}[fragile]
\begin{framed}
\begin{verbatim}
# scatterplots for each combination of two factors 
xyplot(mpg~wt|cyl.f*gear.f, 
  	main="Scatterplots by Cylinders and Gears", 
   ylab="Miles per Gallon", xlab="Car Weight")

# 3d scatterplot by factor level 
cloud(mpg~wt*qsec|cyl.f, 
  	main="3D Scatterplot by Cylinders") 
\end{verbatim}
\end{framed}
\end{frame}
%==============================================%
\begin{frame}[fragile]
\begin{framed}
\begin{verbatim}
# dotplot for each combination of two factors 
dotplot(cyl.f~mpg|gear.f, 
  	main="Dotplot Plot by Number of Gears and Cylinders",
   xlab="Miles Per Gallon")
\end{verbatim}
\end{framed}
\end{frame}
%==============================================%
\begin{frame}[fragile]
\begin{framed}
\begin{verbatim}
# scatterplot matrix 
splom(mtcars[c(1,3,4,5,6)], 
  	main="MTCARS Data")
density1density 2density 3boxplot
scatterplot3D scatterplot dotplotscatterplot matrix
click to view
Note, as in graph 1, that you specifying a conditioning variable is optional. The difference between graphs 2 & 3 is the use of the layout option to contol the placement of panels.
\end{frame}
%==============================================%
\begin{frame}
Customizing Lattice Graphs

Unlike base R graphs, lattice graphs are not effected by many of the options set in the par( ) function. To view the options that can be changed, look at help(xyplot). It is frequently easiest to set these options within the high level plotting functions described above. Additionally, you can write functions that modify the rendering of panels. 
\end{frame}
%==============================================%
\begin{frame}
Here is an example.
# Customized Lattice Example
library(lattice)
panel.smoother <- function(x, y) {
  panel.xyplot(x, y) # show points 
  panel.loess(x, y)  # show smoothed line 
}
\end{verbatim}
\end{framed}
\end{frame}
%==============================================%
\begin{frame}[fragile]
\begin{framed}
\begin{verbatim}
attach(mtcars)
hp <- cut(hp,3) # divide horse power into three bands 
xyplot(mpg~wt|hp, scales=list(cex=.8, col="red"),
  	panel=panel.smoother,
   xlab="Weight", ylab="Miles per Gallon", 
   main="MGP vs Weight by Horse Power")
custom trellis graph click to view
\end{frame}
%==============================================%
\begin{frame}
Going Further
Lattice graphics are a comprehensive graphical system in their own right. Deepanyan Sarkar's book Lattice: Multivariate Data Visualization with R is the definitive reference. Additionally, see the Trellis Graphics homepage and the Trellis User's Guide. Dr. Ihaka has created a wonderful set of slides on the subject. An excellent early consideration of trellis graphs can be found in W.S. Cleavland's classic book Visualizing Data.
\end{verbatim}
\end{framed}
\end{frame}
%==============================================%
\end{document}
