

% - http://www.r-bloggers.com/visualization-of-probabilistic-forecasts/

\documentclass{beamer}
\usepackage{default}
\begin{document}
%============================================================%
%  ESRI Talk 

AER
10 Packages
Forecast
ggplot2
Flood Risk
Survey
sdc


The forecast package for R provides methods and tools for displaying and analysing univariate time series forecasts including exponential smoothing via state space models and automatic ARIMA modelling. It also includes a handful of data sets from the Time Series Data Library. The package is described in Hyndman and Khandakar (2008

%============================================================%

% Statistical Discloure Control
% - http://www.tdp.cat/issues/tdp.a004a08.pdf
Microdata Library Home



microdata.worldbank.org/


In the study of survey and census data, microdata is information at the level of individual respondents.[1] For instance, a national census might collect age, home address, educational level, employment status, and many other variables, recorded separately for every person who responds; this is microdata.



Statistical Disclosure Control for Microdata
Using the R-Package sdcMicro
Matthias Templ

Department of Methodology, Statistics Austria, Guglgasse 13, 1110 Vienna, Austria. Department of Statistics
and Probability Theory, Vienna University of Technology, Wieder Hauptstr. 8-10, 1040 Vienna, Austria.

%============================================================%
\begin{frame}[fragile]
\frametitle{\texttt{R} in Economic and Social Research}

\begin{itemize}
\item The Official Statistics and Survey Methodology Taskview
\end{itemize}

\end{frame}

%============================================================%
\begin{frame}[fragile]

\begin{framed}
\begin{verbatim}

install.packages("pkg1","pkg2")
library(pkg1)
library(pkg2)

\end{verbatim}
\end{framed}


%============================================================%

What is the difference between \texttt{require()} and library()?

There's not much of one in everyday work.

However, according to the documentation for both functions (accessed by putting a ? before the function name and hitting enter), require is used inside functions, as it outputs a warning and continues if the package is not found, whereas library will throw an error.

So you can use require() in constructions like the one below. Which mainly handy if you want to distribute your code to our R installation were packages might not be installed.
%--------------------------------------------------%
\begin{framed}
\begin{verbatim}
if(require("lme4")){
    print("lme4 is loaded correctly")
} else {
    print("trying to install lme4")
    install.packages("lme4")
    if(require(lme4)){
        print("lme4 installed and loaded")
    } else {
        stop("could not install lme4")
    }
}

\end{verbatim}
\end{framed}
%--------------------------------------------------%
?library
and you will see:


library(package) and require(package) both load the package with name package and put it on the search list. require is designed for use inside other functions; it returns FALSE and gives a warning (rather than an error as library() does by default) if the package does not exist. Both functions check and update the list of currently loaded packages and do not reload a package which is already loaded. (If you want to reload such a package, call detach(unload = TRUE) or unloadNamespace first.) If you want to load a package without putting it on the search list, use requireNamespace.

%--------------------------------------------------%

You can use require() if you want to install packages if and only if necessary, such as:
if (!(require(package, character.only=T, quietly=T))) {
    install.packages(package)
    library(package, character.only=T)
}

%============================================================%


%===============================================================%
\begin{frame}[fragile]
\frametitle{Using \texttt{R} for Economic and Social Research}

\textbf{Anonymisation}

\begin{itemize}
\item Compute a frequency table for each variable.
\item Determine if there are any "small cells" (n < 30).
\item Combine Categories
\end{itemize}

\end{frame}
%===============================================================%
\begin{frame}[fragile]
\frametitle{Using \texttt{R} for Economic and Social Research}


\begin{itemize}
\item Determine the smallest "level" for each category
\item If the smallest cell size is still rather large, disregard it.(13\% of population are Left handed). 
\item Set a threshold for crticially small size.
\item Brute Force: For each case, determine how many times they are members of the critically small cells.
\item
\end{itemize}

%===============================================================%
\begin{frame}[fragile]
\frametitle{Using \texttt{R} for Economic and Social Research}

\begin{frame}

\begin{itemize}
\item Econometrics
\item Energy
\item Flood Risk Modelling
\end{itemize}

\end{frame}

%===============================================================%
\begin{frame}[fragile]
\frametitle{Using \texttt{R} for Economic and Social Research}
\begin{itemize}
\textbf{Econometrics}
\begin{itemize}
\item Applied Econometrics with \texttt{R}.
\end{itemize}
\end{frame}

%===============================================================%
\begin{frame}[fragile]

\begin{itemize}
\item \textbf{\textit{foreign}} 
\item \textbf{\textit{ggplot2}} 
\item \textbf{\textit{lme4}}, \textbf{nlme}}
\item \textbf{\textit{vgam}}, \textbf{\textit{nnet}}, \textbf{\textit{MASS}}. 
\item \textbf{\textit{data.table}} and \textbf{\textit{dplyr}} 
\end{itemize}

\end{frame}
%===============================================================%
\begin{frame}[fragile]
\begin{itemize}

\begin{framed}
\begin{verbatim}
libary(gcookbook)

\end{verbatim}
\end{framed}
\end{frame}
%===============================================================%
\begin{frame}[fragile]
\begin{itemize}


