\documentclass{article}
% http://www.dummies.com/how-to/content/how-to-use-regular-expressions-in-r.html
\usepackage[english]{babel}
\usepackage{framed}
\usepackage{times}

\topmargin = 0pt
\marginparwidth = 6pt
\textwidth = 445pt
\oddsidemargin = 10pt
\textheight = 632pt

\title{Introduction to R Programming - Part B}
\date{\vspace{-5ex}}
\begin{document}
\maketitle
\tableofcontents
\section{Data Entry, Data Import and Export}

The scan() function is a useful method of inputting data quickly. You can use to quickly copy and paste values into the R environment.
It is best used in the manner as described in the following example.  Create a variable “X” and use the scan() function to populate it with values.
Type in a value, and then press return.
Once you have entered all the values, press return again to return to normal operation.
\begin{verbatim}
> X=scan()
1: 4
2: 5
3: 5
4: 6
5: 
Read 4 items

\section{Spreadsheet Interface}
R provides a spreadsheet interface for editing the values of existing data sets.
We use the command data.entry() , and name of the data object as the argument.
> data.entry(X) # Edit the data set and exit interface
> X

 

\section{Data Import}
It is necessary to import outside data into R before you start analysing it
Quite often, the sample data is in Excel format, and needs to be imported into R prior to use. For this, we use the read.xls() function from the “gdata” package. It reads from an Excel spreadsheet and returns a data frame. The following shows how to load an Excel spreadsheet named "mydata.xls". As the package is not in the core R library, it has to be installed and loaded into the R workspace.
\begin{verbatim}
> library(gdata)                   # load the gdata package 
> help(read.xls)                   # documentation 
> mydata = read.xls("mydata.xls")  # read from first sheet
\end{verbatim}

\section{Table File}
 A data table can resides in a text file. The cells inside the table are separated by blank characters. Here is an example of a table with 4 rows and 3 columns.
100   a1   b1 
200   a2   b2 
300   a3   b3 
400   a4   b4 
 
 
Now copy and paste the table above in a file named "mydata.txt" with a text editor. Then load the data into the workspace with the read.table() function.
> mydata = read.table("mydata.txt")  # read text file 
> mydata                             # print data frame 
   V1 V2 V3 
1 100 a1 b1 
2 200 a2 b2 
3 300 a3 b3 
4 400 a4 b4 

> help(read.table) 


\section{Data export}
The basic tool to produce output files is write.table().
 
The only required argument to write.table()is the name of a dataset or matrix; with just a single argument, the output will be printed on the console, making it easy to test that the file you’ll be creating is in the correct format. 

Usually, the second argument, file= will be used to specify the destination as either a character string to represent a file, or a connection (i.e. database connectivity).

By default, character strings are surrounded by quotes by write.table(); use the quote=FALSE argument to suppress this feature. To suppress row names or column names from being written to the file, use the row.names=FALSE or col.names=FALSE arguments, respectively. 

Note that col.names=TRUE (the default) produces the same sort of headers that are read using the header=TRUE argument of read.table(). 

Finally, the sep= argument can be used to specify a separator other than a blank space. Using sep=’,’ (comma separated) or sep=’\t’ (tab-separated) are two common choices.


write.table(CO2 ,file=’co2.txt’, row.names=FALSE, sep=’,’)


>
Similarly to read.csv and read.csv2, the functions write.csv and write.csv2 are provided as wrappers to read.table, with appropriate options set to produce comma- or semicolon-separated files. 

%--------------------------------------------------------------------------------------%
\section{CSV File}
The sample data can also be in comma separated values (CSV) format. Each cell inside such data file is separated by a special character, which usually is a comma, although other characters can be used as well.
The first row of the data file should contain the column names instead of the actual data. Here is a sample of the expected format.

\begin{verbatim}
Col1,Col2,Col3 
100,a1,b1 
200,a2,b2 
300,a3,b3 
\end{verbatim}

After we copy and paste the data above in a file named "mydata.csv" with a text editor, we can read the data with the read.csv function.
> mydata = read.csv("mydata.csv")  # read csv file 
> mydata                           # print data frame 
  Col1 Col2 Col3 
1  100   a1   b1 
2  200   a2   b2 
3  300   a3   b3 
> help(read.csv)

In various European locales, as the comma character serves as decimal point, the read.csv2 function should be used instead.
%--------------------------------------------------------------------------------------%
\end{document}

