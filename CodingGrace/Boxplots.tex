\documentclass{beamer}

\usepackage{graphics}

\begin{document}

\begin{frame}

{\Large
\\
\begin{center}
\vspace{1.3cm}
Constructing Boxplots with \texttt{R}.\\
\vspace{1.3cm}
{\large kobriendublin.wordpress.com}
\end{center}
}


\end{frame}

\begin{frame}
\frametitle{Boxplots}
{
\Large
\vspace{-0.7cm}
\textbf{Components of a boxplot}\\
\vspace{0.4cm}
The location and shape of the central component (i.e. the ``Box") is determined by the following values:
\begin{itemize}
\item First Quartile $Q_1$
\item Median $\tilde{x}$
\item Third Quartile $Q_3$
\end{itemize}
}
\end{frame}


\begin{frame}
\frametitle{Boxplots}
{
\Large
\textbf{Components of a boxplot}\\
\vspace{0.4cm}
The rest of the plot, including the ``Whiskers", are determined by the following:
\vspace{0.6cm}
\begin{itemize}
\item The presence of outliers
\begin{itemize}
\item Any value greater than $Q_3 + 1.5 \times IQR $
\item Any value less than $Q_1 - 1.5 \times IQR $
\end{itemize}
\item Largest Value (that is not an outlier).
\item Smallest Value (that is not an outlier).
\end{itemize}
}
\end{frame}

\end{document}
