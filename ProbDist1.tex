Section 6: Simulation and Probability Distributions
Probability Distributions
  
Distribution
R name
Additional Arguments
beta
beta
shape1, shape2, ncp
binomial
binom
size, prob
Cauchy
cauchy
location, scale
Chi-squared
chisq
df, ncp
exponential
exp
rate
F
f
df1, df2
geometric
geom
prob
hypergeometric
hyper
values m,n,k
normal
norm
mean,sd
Poisson
pois
lambda
Student's
r
df,ncp
uniform
unif
min,max
Weibull
weibul
shape,scale
Wilcoxon
wilcox
m,n






Probability
 Simulation Study: Simulating dice throws

  


X = runif(100,0,6)

%================================================================================================ %
Discrete Probability Distributions

The two most accessible discrete distributions are the binomial and poisson distributions
The binomial distribution is characterized by the number of trials, which in R is denoted as ‘size’ rather than ‘n’, and the probability of success ‘prob’.
rbinom(n=5,size=100,p=0.25)		#generate five random numbers

The Poisson distribution is characterized the by the arrival rate ‘lambda’.
rpois(n=5,lambda=4)			#generate five random numbers

Simple population study.

Suppose a small island has population 1,000 at the start of a decade. The birth rate on this island is expected to 25 births per annum, while there is on average 10 deaths.  Forecast the population after five years.

Base = 1000
Births =rpois(5,25)
Deaths=rpois(5,10)
Yrly.Incr =Births - Deaths
Increase =cumsum(Yrly.Incr)
Popn = Base + Births +Deaths

Continuous Probability Distributions]

The continuous uniform distribution is commonly used in simulation.
The Normal distribution
The normal distribution is perhaps the most w
rnorm(n=15)				#15 random numbers, mean  = 0 , std. deviation = 1
rnorm(n=15,mean= 17)			#set the mean to 17
rnorm(n=15,mean= 17,sd=4)		#set the standard deviation to 4
rnorm(15,17,4)				#argument matching : default positions


The exponential distribution

%============================================================================================================== %