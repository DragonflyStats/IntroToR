
\subsection*{ifelse Statement}
A vectorised version of the if statement is ifelse. This is useful
if you want to perform some action on every element of a vector
that satis es some condition.
The syntax is
ifelse( condition, true expr, false expr )
If condition == TRUE, the true expr is carried out. If
condition == FALSE, the false expr is carried out.
\begin{framed}
\begin{verbatim}
x <- rnorm(20, mean=15, sd=5)
x
[1] 23.608513 14.424667 12.306040 14.291568 18.522846 14.514071 22.004400
[8] 24.658249 11.697999 16.344976 22.110389 8.455789 19.672274 22.393680
[15] 11.449034 17.288859 14.839597 14.484774 18.636589 22.670548
ifelse(x >= 17, sqrt(x), NA)
[1] 4.858859 NA NA NA 4.303818 NA 4.690885
[8] 4.965707 NA NA 4.702169 NA 4.435344 4.732196
[15] NA 4.157987 NA NA 4.317012 4.761360
    
\end{verbatim}
\end{framed}


\subsection*{for Loops}
Repetitive execution: for loops, while loops and repeat loops.
To loop/iterate through a certain number of repetitions a for loop
is used. The basic syntax is
\begin{verbatim}
for(variable_name in sequence) {
command
command
command
}
\end{verbatim}

A simple example of a for loop is:
\begin{verbatim}
for(i in 1:5){
print(sqrt(i))
}
[1] 1
[1] 1.414214
[1] 1.732051
[1] 2
[1] 2.236068
\end{verbatim}

%%%%%%%%%%%%%%%%%%%%%%%%%%%%%%%%%%5
Another example is:
\begin{verbatim}
n <- 20
p <- 5
value <- vector(mode="numeric", length=n)
rand.nums <- matrix(rnorm(n*p), nrow=n)
for(i in 1:length(value)){
value[i] <- max(rand.nums[i,])
print(sum(value))
}
\end{verbatim}
The first four lines create variables n and p with values 20 and 5
respectively, a numeric vector called value with length 20 and a
matrix of 20*5=100 random numbers, called rand.nums, with 20
rows.
The for loop performs 20 loops and stores the maximum value
from each row of rand.nums into position i of the vector value.
The sum of the current numbers in value is also printed to the
screen.

%%%%%%%%%%%%%%%%%%%%%%%%%%%%%%%%%%%%%%%%%%%%%%%%%%%%%%%
Can also have nested for loops. Indenting your code can be useful
when trying to \match" brackets.
for(variable_name1 in sequence) {
command
command
for(variable_name2 in sequence) {
command
command
command
} # ends inner for loop
} # ends outer for loop
It should be noted that variable_name2 should be di erent from
variable_name1, e.g. use i and j. Using the same name will
reset the counter each time and result in an in nite loop!!


%%%%%%%%%%%%%%%%%%%%%%%%%%%%%%%%%%%%%%%%%%%%%%%%%%%%%%%
Load the function simple.nesting from Loops.R and call the
function using
simple.nesting(num.fam=5, num.child=3).
The le nest.dat will be created in your current working
directory. Open this file and explore the contents.
for loops and multiply nested for loops are generally avoided
when possible in R as they can be quite slow. We will use in
simulation examples later in the course.

%%%%%%%%%%%%%%%%%%%%%%%%%%%%%%%%%%%%%%%%%%%%%%%%%%%%%%%

The while loop can be used if the number of iterations required is
not known beforehand. For example, if we want to continue
looping until a certain condition is met, a while loop is useful.
The following is the syntax for a while loop:
\begin{verbatim}
while (condition){
command
command
}
The loop continues while condition == TRUE.
niter <- 0
num <- sample(1:100, 1)
while(num != 20) {
num <- sample(1:100, 1)
niter <- niter + 1
}
niter
    
\end{verbatim}

%%%%%%%%%%%%%%%%%%%%%%%%%%%%%%%%%%%%%%%%%%%%%%%%%%%%%%%
\subsection*{next, break, repeat Statements}
The next statement can be used to discontinue one particular
iteration of any loop, i.e. this iteration is ended and the loop
\skips" to the next iteration. Useful if you want a loop to continue
even if an error is found (error checking).
The break statement completely terminates a loop. Useful if you
want a loop to end if an error is found. See the Loops.R script le
for code to exhibit the di erence between the next and break
statements.
The repeat loop uses next and break. The only way to end this
type of loop is to use the break statement. For an example, see
the Loops.R script file.
\end{document}
