\documentclass{beamer}

% The scale function
% The dist Function
% coef and resid
% Confindence Intervals for regression Coefficients
% Standardized Regression coefficients
\usepackage{framed}
\usepackage{amsmath}
\usepackage{amssymb}
\begin{document}
\begin{frame}
\Huge
\[\mbox{Computing with \texttt{R}}\]
\[\mbox{The \texttt{scale()} function}\]
\bigskip
\LARGE
\[\mbox{www.Stats-Lab.com}\]
\[\mbox{Twitter: @StatsLabDublin}\]
\end{frame}
%-------------------------------------%
\begin{frame}[fragile]
\frametitle{Computing with \texttt{R} : The \texttt{scale()} function}
\Large
\vspace{-1cm}
\textbf{Data:} \\
The \texttt{X} and \texttt{Y} variables are the $wt$ and $mpg$ variables from the \textit{\textbf{\textit{mtcars}}} data set.

\begin{framed}
\begin{verbatim}
X <- mtcars$wt
Y <- mtcars$mpg
\end{verbatim}
\end{framed}

\end{frame}
%--------------------------------------%
% Set 1 : The Scale Function
%--------------------------------------%
\begin{frame}
\frametitle{Computing with \texttt{R} : The \texttt{scale()} function}
\Large
\vspace{-1cm}
\begin{itemize}
\item The scale function is used to determine standardized values for each element in a data set.
\vspace{0.2cm}
\item This is a data transformation technique that can be used in regression and clustering analysis.
\end{itemize}
\end{frame}
%--------------------------------------%
% Set 1 : The Scale Function
%--------------------------------------%
\begin{frame}
\frametitle{Computing with \texttt{R} : The \texttt{scale()} function}
\Large
\begin{itemize}
\item A standardized value for an element is simply the number of standard deviations away from the mean.
\item Suppose $z_i$ is the standardized value for $x_i$, an element of a sample data set with mean $\bar{x}$ and standard deviaton $s$.
\[z_i = \frac{x_i-\bar{x}}{s}\]
\end{itemize}
\end{frame}

%--------------------------------------%
\begin{frame}[fragile]
\Large
\begin{verbatim}
.....
[32,] -0.446876870
attr(,"scaled:center")
[1] 3.21725
attr(,"scaled:scale")
[1] 0.9784574
> mean(X)
[1] 3.21725
> sd(X)
[1] 0.9784574
\end{verbatim}
\end{frame}
%--------------------------------------%
% Set 2 : The dist Function
%--------------------------------------%
\begin{frame}
\Huge
\[\mbox{Computing with \texttt{R}}\]
\[\mbox{The \texttt{dist()} function}\]
\bigskip
\LARGE
\[\mbox{www.Stats-Lab.com}\]
\[\mbox{Twitter: @StatsLabDublin}\]
\end{frame}
%-------------------------------------%
\begin{frame}[fragile]
\frametitle{Computing with \texttt{R} : The \texttt{dist()} function}
\Large
\vspace{-1cm}
\textbf{Data:} \\
The first 8 rows and first 5 columns from the \textbf{mtcars} dataset.

\begin{framed}
\begin{verbatim}
X <- mtcars[1:8,1:5]
\end{verbatim}
\end{framed}
\end{frame}
%-------------------------------------%
\begin{frame}[fragile]
\frametitle{Computing with \texttt{R} : The \texttt{dist()} function}
\Large
\vspace{-1cm}

\begin{itemize}
\item The \texttt{dist()} function is used to compute the \textbf{distance matrix}.
\item The distance matrix is comprised of distance measures for each pair of cases in the data set.
\item A distance measure is a measure of \emph{\bf{similarity}} between two cases, based on a set of numeric values.
\end{itemize}
\end{frame}
%-------------------------------------%
\begin{frame}[fragile]
\frametitle{Computing with \texttt{R} : The \texttt{dist()} function}
\Large
\vspace{-1cm}

\begin{itemize}
\item The default distance measure is the \textbf{Euclidean Distance}.
\item Given three numeric variables $X$, $Y$ and $Z$, the Euclidean distance between case 1 and case 2 is computed as
\[ED_{12} =  \sqrt{(x_1-x_2)^2 + (y_1-y_2)^2 + (z_1-z_2)^2}\]
\end{itemize}
\end{frame}
%-------------------------------------%
\begin{frame}[fragile]
\frametitle{Computing with \texttt{R} : The \texttt{dist()} function}
\Large
\vspace{-1cm}
 Other types of distance measure that can be specified are
\begin{itemize}

\item the ``maximum" measure,\item the ``Manhattan" measure,\item the ``Canberra" measure,\item the ``binary" measure \item the ``Minkowski" measure.

\end{itemize}
\end{frame}
%-------------------------------------%
\begin{frame}[fragile]
\frametitle{Computing with \texttt{R} : The \texttt{dist()} function}
\Large
\textbf{Transforming the Data}
\begin{itemize}
\item Sometimes it would beneficial to transform one or more of the variables to prevent them being unduly influential, at the expense of other variables.
\item One approach is to use standardized values. Standardization can be performed using the \texttt{scale()} function.
\item Another approach is logarithmic transformation, which can be performed using the \texttt{log()} function.
\end{itemize}


\end{frame}


%--------------------------------------%
% Set 3 : Useful Functions for Linear Regression
%--------------------------------------%
\begin{frame}
\huge
\[\mbox{Computing with \texttt{R}}\]
\[\mbox{Useful Regression Functions}\]
\bigskip
\LARGE
\[\mbox{www.Stats-Lab.com}\]
\[\mbox{Twitter: @StatsLabDublin}\]
\end{frame}
%-------------------------------------%
%-------------------------------------%
\begin{frame}[fragile]
\frametitle{Useful Functions for Linear Regression}
\Large
\vspace{-1cm}
\textbf{Data:} \\
The \texttt{X} and \texttt{Y} variables are the $wt$ and $mpg$ variables from the \textit{\textbf{\textit{mtcars}}} data set.

\begin{framed}
\begin{verbatim}
X <- mtcars$wt
Y <- mtcars$mpg

Fit <- lm(Y~X)
\end{verbatim}
\end{framed}
\end{frame}
%-------------------------------------%
%-------------------------------------%
\begin{frame}
\frametitle{Useful Functions for Linear Regression}
\Large
\begin{itemize}
\item \texttt{summary()} - very detailed statistical summary of the fitted model,
\item \texttt{coef()} - prints out the regression coefficients for the fitted model,
\item \texttt{fitted(}) - prints out the fitted values for the fitted model,
\item \texttt{resid()} - prints out the residual for the fitted model,
\item \texttt{anova()} - prints out the ANOVA table for the fitted model.
\end{itemize}
\end{frame}
%-------------------------------------%
%-------------------------------------%


%--------------------------------------%
% Set 4 : Useful Functions for Linear Regression
%--------------------------------------%
\begin{frame}
\Huge
\[\mbox{Computing with \texttt{R}}\]
\huge
\[\mbox{Confidence Intervals}\]
\[\mbox{for Regression Coefficients}\]
\bigskip
\LARGE
\[\mbox{www.Stats-Lab.com}\]
\[\mbox{Twitter: @StatsLabDublin}\]
\end{frame}
%-------------------------------------%
%-------------------------------------%
\begin{frame}[fragile]
\frametitle{Confidence Intervals for Regression Coefficients}
\Large
\vspace{-1cm}
\textbf{Data:} \\
The \texttt{X1}, \texttt{X2} and \texttt{Y} variables are the $wt$ , $hp$ and $mpg$ variables from the \textit{\textbf{\textit{mtcars}}} data set.

\begin{framed}
\begin{verbatim}
X1 <- mtcars$wt
X2 <- mtcars$hp
Y <- mtcars$mpg

Fit <- lm(Y~X1+X2)
\end{verbatim}
\end{framed}
\end{frame}
%-------------------------------------%
%-------------------------------------%
\begin{frame}
\frametitle{Confidence Intervals for Regression Coefficients}
\Large
\vspace{-1cm}
\begin{itemize}
\item To compute the confidence intervals, we use the \texttt{confint()} function, specifying the name of the fitted model. \vspace{0.2cm}
\item The default confidence level is 95\%. We can adjust it by changing the \texttt{level=} argument. (e.g. \texttt{level = 0.90}).
\item We can specify the confidence interval for particular regression coefficients using the \texttt{parm=} argument.
\end{itemize}
\end{frame}
%--------------------------------------%
% Set 4 : Useful Functions for Linear Regression
%--------------------------------------%
\begin{frame}
\Huge
\[\mbox{Computing with \texttt{R}}\]
\LARGE
\[\mbox{Standardized Regression Coefficients}\]
\bigskip
\LARGE
\[\mbox{www.Stats-Lab.com}\]
\[\mbox{Twitter: @StatsLabDublin}\]
\end{frame}
%-------------------------------------%
%-------------------------------------%
\begin{frame}[fragile]
\frametitle{Standardized Regression Coefficients}
\Large
\vspace{-1cm}
\textbf{Data:} \\
The \texttt{X} and \texttt{Y} variables are the $wt$ and $mpg$ variables from the \textit{\textbf{\textit{mtcars}}} data set.

\begin{framed}
\begin{verbatim}
X1 <- mtcars$wt
X2 <- mtcars$hp
Y <- mtcars$mpg

Fit.u <- lm(Y ~ X1 + X2)
\end{verbatim}
\end{framed}
\end{frame}
%-------------------------------------%
%-------------------------------------%

\begin{frame}[fragile]
\frametitle{Standardized Regression Coefficients}
\Large
\vspace{-1cm}

\begin{itemize}
\item In some statistical analyses, it is useful to work with standardized values, rather than observed values.
\item A standardized value for an element is simply the number of standard deviations away from the mean.
\end{itemize}
\end{frame}
%-------------------------------------%
%-------------------------------------%

\begin{frame}[fragile]
\frametitle{Standardized Regression Coefficients}
\Large
\vspace{-1cm}

\begin{itemize}
\item To compute a regression model on standardized values, use the \texttt{scale()} function to standardize all of the relevant variables.
\end{itemize}
\begin{framed}
\begin{verbatim}
Fit.s <- lm( scale(Y) ~ scale(X1) +
  scale(X2) )
\end{verbatim}
\end{framed}
\end{frame}
%-------------------------------------%
%-------------------------------------%

\end{document}
