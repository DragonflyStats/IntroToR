%====================================================================================================%
\section{Factors}
A factor is a special type of vector used to represent categorical
data, e.g. gender, social class, etc.
\begin{itemize}
	\item Stored internally as a numeric vector with values 1, 2, : : : ; k,
	where k is the number of levels.
	\item Can have either ordered and unordered factors.
	\item A factor with k levels is stored internally consisting of 2 items
	\begin{itemize}
		\item[(a)] a vector of k integers
		\item[(b)] a character vector containing strings describing what the k
		levels are.
	\end{itemize}
	
	
	%===========================================================================================%
	\begin{frame}
		
		\subsection{Factors: Example}
		Consider a survey that has data on 200 females and 300 males. If
		the rst 200 values are from females and the next 300 values are
		from males, one way of representing this is to create a vector
		\begin{framed}
			\begin{verbatim}
			> gender <- c(rep("female", 200), rep("male", 300))
			> #To change this into a factor
			> gender <- factor(gender)
			>
			\end{verbatim}
		\end{framed}
		The factor gender is stored internally as
		\begin{enumerate}
	\item female
	\item male
		\end{enumerate}
	
		Each category, i.e. female and male, is called a level of the factor.
		To determine the levels of a factor the function levels() can be
		used:
		\begin{framed}
			\begin{verbatim}
			> levels(gender)
			[1] "female" "male"
			\end{verbatim}
		\end{framed}
		%-------------------------------------------------------%
		\subsection{Factors: Example}
		Five people are asked to rate the performance of a product on a
		scale of 1-5, with 1 representing very poor performance and 5
		representing very good performance. The following data were
		collected.
		\begin{framed}
			\begin{verbatim}
			> satisfaction <- c(1, 3, 4, 2, 2)
			> fsatisfaction <- factor(satisfaction, levels=1:5)
			> levels(fsatisfaction) <- c("very poor", "poor", "average",
			"good", "very good")
			\end{verbatim}
		\end{framed}
		%-------------------------------------------------------%
		\begin{itemize}
			\item The first line creates a numeric vector containing the satisfaction
			levels of the 5 people. Want to treat this as a categorical variable
			and so the second line creates a factor. 
			\item The \texttt{levels=1:5} argument
			indicates that there are 5 levels of the factor. Finally the last line
			sets the names of the levels to the specified character strings.
		\end{itemize}
		%===========================================================================================%
	\section{Vector Arithmetics}

Recall that the vector is a sequence of data elements of the same basic type. (Members in a vector are officially called components.)

Arithmetic operations of vectors are performed member-by-member, i.e., memberwise. For example, suppose we have two vectors a and b.

> a = c(1, 3, 5, 7) 
> b = c(1, 2, 4, 8)

Then, if we multiply a by 5, we would get a vector with each of its members multiplied by 5.

> 5 * a 
[1]  5 15 25 35

And if we add a and b together, the sum would be a vector whose members are the sum of the corresponding members from a and b.

> a + b 
[1]  2  5  9 15

Similarly for subtraction, multiplication and division, we get new vectors via memberwise operations.

> a - b 
[1]  0  1  1 -1 
> a * b 
[1]  1  6 20 56 
> a / b 
[1] 1.000 1.500 1.250 0.875

Recycling Rule
If two vectors are of unequal length, the shorter one will be recycled in order to match the longer vector. For example, the following vectors u and v have different lengths, and their sum is computed by recycling values of the shorter vector u.

> u = c(10, 20, 30) 
> v = c(1, 2, 3, 4, 5, 6, 7, 8, 9) 
> u + v 

Empty Vectors

Often we would require to populate a vector dynamically ( specifically simulation studies) . But before such an operation takes place,  a variable must be created and structured, according to the type of operation. Once the operation commences, the vector is populated. 

To construct an “empty” vector, we use the command numeric(). Alternatively we may require a numeric vector of specified length, initially populated with zeroes.

X =  numeric()
length(X)
## Length 0

X2 =  numeric(2)
## 0 0 

Later, we are going to use this function in computing the first 50 terms of Fibonacci Sequence.

Similarly we can create “empty vectors” for logical and character data structures.

Y=logical()
Z=character()

Vector Index

We retrieve values in a vector by declaring an index inside a single square bracket "[]" operator.

For example, the following shows how to retrieve a vector member. Since the vector index is 1-based, we use the index position 3 for retrieving the third member.

> s = c("aa", "bb", "cc", "dd", "ee") 
> s[3] 
[1] "cc"

Unlike other programming languages, the square bracket operator returns more than just individual members. In fact, the result of the square bracket operator is another vector, and s[3] is a vector slice containing a single member "cc".

Negative Index
If the index is negative, it would strip the member whose position has the same absolute value as the negative index. For example, the following creates a vector slice with the third member removed.

> s[-3] 
[1] "aa" "bb" "dd" "ee"

Range Index
To produce a vector slice between two indexes, we can use the colon operator ":". This is the same method that was used to computing a sequence of integers. This can be convenient for situations involving large vectors.

> s[2:4] 
[1] "bb" "cc" "dd"

Out-of-Range Index
If an index is out-of-range, a missing value will be reported via the symbol NA.

> s[10] 
[1] NA

Named Vector Members

We can assign names to vector members. For example, the following variable v is a character string vector with two members.

> v = c("LouLou", "Oscar") 
> v 
[1] "LouLou" "Oscar"

We now name the first member as One, and the second as Two.


> names(v) = c("One", "Two") 
> v 
 One  Two 
"LouLou"  "Oscar"


Then we can retrieve the first member by its name.


> v["One"] 
[1] "LouLou"

Furthermore, we can reverse the order with a character string index vector.

> v[c("Two", "One")] 
  Two  One 
 "LouLou" "Oscar"









%----------------------------------------------------------------------------------------------------%
\documentclass[KSmain.tex]{subfiles} 
\begin{document} 

\section{Data Structures}

pandas introduces two new data structures to Python - Series and DataFrame, both of which are built on top of NumPy (this means it's fast).

\begin{framed}
\begin{verbatim}
import pandas as pd
import numpy as np
import matplotlib.pyplot as plt
pd.set_option('max_columns', 50)
\end{verbatim}
\end{framed}
%-----------------------------------------%
\subsection{Series}

Series is a one-dimensional labeled array capable of holding any data type (integers, strings, floating point numbers, Python objects, etc.). The axis labels are collectively referred to as the index. The basic method to create a Series is to call:
 
\begin{framed}
\begin{verbatim}
s = Series(data, index=index)
\end{verbatim}
\end{framed} 
Here, data can be many different things:
 
\begin{itemize}
\item a Python dict
\item an ndarray
\item a scalar value (like 5)
\end{itemize}

%-----------------------------------------%
A Series is a one-dimensional object similar to an array, list, or column in a table. It will assign a labeled index to each item in the Series. By default, each item will receive an index label from 0 to N, where N is the length of the Series minus one.

\begin{framed}
\begin{verbatim}
# create a Series with an arbitrary list
s = pd.Series([7, 'Heisenberg', 3.14, -1789710578, 'Happy Eating!'])
s
\end{verbatim}
\end{framed}
\begin{verbatim}
0                7
1       Heisenberg
2             3.14
3      -1789710578
4    Happy Eating!
dtype: object
\end{verbatim}

Alternatively, you can specify an index to use when creating the Series.

\begin{framed}
\begin{verbatim}
s = pd.Series([7, 'Heisenberg', 3.14, -1789710578, 'Happy Eating!'],
              index=['A', 'Z', 'C', 'Y', 'E'])
s
\end{verbatim}
\end{framed}
\begin{verbatim}
A                7
Z       Heisenberg
C             3.14
Y      -1789710578
E    Happy Eating!
dtype: object
\end{verbatim}

The Series constructor can convert a dictonary as well, using the keys of the dictionary as its index.

\begin{framed}
\begin{verbatim}
d = {'Chicago': 1000, 'New York': 1300, 'Portland': 900, 'San Francisco': 1100,
     'Austin': 450, 'Boston': None}
cities = pd.Series(d)
cities
Out[4]:
Austin            450
Boston            NaN
Chicago          1000
New York         1300
Portland          900
San Francisco    1100
dtype: float64
\end{verbatim}
\end{framed}
You can use the index to select specific items from the Series ...

\begin{framed}
\begin{verbatim}
cities['Chicago']
Out[5]:
1000.0
\end{verbatim}
\end{framed}

\begin{framed}
\begin{verbatim}
cities[['Chicago', 'Portland', 'San Francisco']]
Out[6]:
Chicago          1000
Portland          900
San Francisco    1100
dtype: float64
\end{verbatim}
\end{framed}
Or you can use boolean indexing for selection.

\begin{framed}
\begin{verbatim}
cities[cities < 1000]
Out[7]:
Austin      450
Portland    900
dtype: float64
\end{verbatim}
\end{framed}
That last one might be a little strange, so let's make it more clear - \texttt{cities < 1000} returns a Series of \texttt{True/False} values, which we then pass to our Series cities, returning the corresponding \texttt{True} items.

\begin{framed}
\begin{verbatim}
less_than_1000 = cities < 1000
print less_than_1000
print '\n'
print cities[less_than_1000]
Austin            True
Boston           False
Chicago          False
New York         False
Portland          True
San Francisco    False
dtype: bool


Austin      450
Portland    900
dtype: float64

\end{verbatim}
\end{framed}
You can also change the values in a Series on the fly.

\begin{framed}
\begin{verbatim}
# changing based on the index
print 'Old value:', cities['Chicago']
cities['Chicago'] = 1400
print 'New value:', cities['Chicago']
Old value: 1000.0
New value: 1400.0
\end{verbatim}
\end{framed}

\begin{framed}
\begin{verbatim}
# changing values using boolean logic
print cities[cities < 1000]
print '\n'
cities[cities < 1000] = 750

print cities[cities < 1000]
Austin      450
Portland    900
dtype: float64


Austin      750
Portland    750
dtype: float64
\end{verbatim}
\end{framed}
What if you aren't sure whether an item is in the Series? You can check using idiomatic Python.

\begin{framed}
\begin{verbatim}
print 'Seattle' in cities
print 'San Francisco' in cities
False
True
\end{verbatim}
\end{framed}
Mathematical operations can be done using scalars and functions.

\begin{framed}
\begin{verbatim}
# divide city values by 3
cities / 3
Out[12]:
Austin           250.000000
Boston                  NaN
Chicago          466.666667
New York         433.333333
Portland         250.000000
San Francisco    366.666667
dtype: float64

\begin{framed}
\begin{verbatim}
# square city values
np.square(cities)
Out[13]:
Austin            562500
Boston               NaN
Chicago          1960000
New York         1690000
Portland          562500
San Francisco    1210000
dtype: float64
\end{verbatim}
\end{framed}
You can add two Series together, which returns a union of the two Series with the addition occurring on the shared index values. Values on either Series that did not have a shared index will produce a NULL/NaN (not a number).

\begin{framed}
\begin{verbatim}
print cities[['Chicago', 'New York', 'Portland']]
print'\n'
print cities[['Austin', 'New York']]
print'\n'
print cities[['Chicago', 'New York', 'Portland']] + cities[['Austin', 'New York']]
\end{verbatim}
\end{framed}
\begin{verbatim}
Chicago     1400
New York    1300
Portland     750
dtype: float64


Austin       750
New York    1300
dtype: float64


Austin       NaN
Chicago      NaN
New York    2600
Portland     NaN
dtype: float64
\end{verbatim}

Notice that because Austin, Chicago, and Portland were not found in both Series, they were returned with NULL/NaN values.

NULL checking can be performed with \texttt{isnull} and \texttt{notnull}.
%----------------------------------------------------------------%
\begin{framed}
\begin{verbatim}
# returns a boolean series indicating which values aren't NULL
cities.notnull()
Out[15]:
Austin            True
Boston           False
Chicago           True
New York          True
Portland          True
San Francisco     True
dtype: bool
In [16]:
# use boolean logic to grab the NULL cities
print cities.isnull()
print '\n'
print cities[cities.isnull()]
Austin           False
Boston            True
Chicago          False
New York         False
Portland         False
San Francisco    False
dtype: bool


Boston   NaN
dtype: float64
\end{verbatim}
\end{framed}

\end{document}

%%---------------------------------------%
%\newpage
%\subsection{DataFrame}
%
%% pandas - chapter 5 - DataFrame
%
%A DataFrame is a tablular data structure comprised of rows and columns, akin to a spreadsheet, database table, or R's data.frame object. You can also think of a DataFrame as a group of Series objects that share an index (the column names).
%
%%For the rest of the tutorial, we'll be primarily working with DataFrames.
%
%%---------------------------------------%
%\newpage
%\subsection{Panel}
%
%
% 
%\texttt{Panel} is a somewhat less-used, but still important container for 3-dimensional data. 
%The term panel data is derived from econometrics and is partially responsible for the name pandas: pan(el)-da(ta)-s. 
%The names for the 3 axes are intended to give some semantic meaning to describing operations involving panel data and, 
%in particular, econometric analysis of panel data. However, for the strict purposes of slicing and dicing a 
%collection of DataFrame objects, you may find the axis names slightly arbitrary:
% 
%\begin{itemize}
%\item items: axis 0, each item corresponds to a DataFrame contained inside
%\item major\_axis: axis 1, it is the index (rows) of each of the DataFrames
%\item minor\_axis: axis 2, it is the columns of each of the DataFrames
%\end{itemize}
%
%\newpage
%
%\end{document}
