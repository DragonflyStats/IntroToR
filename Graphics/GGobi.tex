GGobi is an open source visualization program for exploring high-dimensional data.
It is freely available for MS Windows, Linux, and Mac platforms. It supports linked interactive scatterplots, barcharts, parallel coordinate plots and tours, with both brushing and identification. 
A good tutorial is included with the GGobi manual. You can download the software here.
http://visweek.org/visweek/2012/session/tutorial-visualizing-data-r-and-ggobi
GGobi is a software tool for dynamic data exploration, allowing you to combine different visualization types like scatterplots and parallel coordinates. It is especially designed for multidimensional datasets. By using the programming language R you can automatize the visualization creation.
GGobi is a free statistical software tool used for graphing various types of data. GGobi allows extensive exploration of the data with Interactive dynamic graphics. It is also a tool for looking at multivariate data. R can be used in sync with GGobi (through rggobi). GGobi prides itself on its ability to link multiple graphs together.[2]

